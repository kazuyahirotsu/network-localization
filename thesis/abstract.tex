\documentclass[12pt,a4paper]{article}
\usepackage[utf8]{inputenc}
\usepackage[T1]{fontenc}
\usepackage{amsmath, amssymb}
\usepackage{geometry}
\geometry{top=25mm, bottom=25mm, left=25mm, right=25mm}

\title{\textbf{Resilient UAV Navigation in GNSS-Denied Environments via Physics-Aware GCN and Robust Multilateration}}
\author{Kazuya Hirotsu}
\date{}

\begin{document}

\maketitle

\section*{Abstract}

\subsection*{1. Introduction}
The deployment of Unmanned Aerial Vehicles (UAVs) has transformed critical sectors such as defense, infrastructure inspection, and disaster response.
However, the operational reliability of these autonomous systems remains heavily dependent on Global Navigation Satellite Systems (GNSS) like GPS.
This reliance constitutes a significant vulnerability: GNSS signals are prone to obstruction in urban canyons and mountainous terrain, and are increasingly susceptible to intentional jamming and spoofing in conflict zones.
While alternative navigation technologies such as Visual SLAM (Simultaneous Localization and Mapping) and LiDAR have advanced, they impose significant computational burdens, require expensive hardware, and fail in feature-poor environments like deserts, snow-covered terrain, or over water.
Thus, there is a critical need for a resilient, cost-effective, and rapidly deployable backup navigation system that functions independently of GNSS.

This thesis investigates the feasibility of using Low-Power Wide-Area (LPWA) networks, specifically LoRa (Long Range), as a standalone positioning infrastructure for GNSS-denied environments.
LoRa offers distinct advantages for large-scale deployment, including kilometer-range communication, low power consumption, and minimal hardware cost.
However, repurposing LoRa for localization presents two fundamental challenges.
First, on the client side, localization based on Received Signal Strength Indicator (RSSI) is inherently noisy due to multipath fading and environmental shadowing, which are difficult to model in complex terrain.
Second, on the infrastructure side, deploying a localization network in a GNSS-denied disaster zone creates a fundamental initialization challenge: the beacons (anchors) themselves must first determine their own positions before they serve as reference points for drones.

In this thesis, we propose a comprehensive ``Resilient Navigation Infrastructure'' architecture that addresses these challenges through a two-phase approach: (1) Robust Client-Side Localization, which secures drone positioning assuming known infrastructure, and (2) Infrastructure Self-Organization, which enables the beacon network to localize itself in unknown environments.
By integrating these two phases, we demonstrate a survivable navigation system capable of guiding UAVs through GNSS-denied regions with operationally feasible accuracy.
Specifically, our contributions are three-fold:
\begin{enumerate}
    \item We introduce a robust multilateration algorithm that integrates altitude constraints and a Huber loss function, significantly reducing the impact of RSSI outliers on drone positioning.
    \item We develop a Physics-Aware Graph Convolutional Network (GCN) that allows a sparse network of beacons to self-localize in complex terrain by learning a terrain-aware path-loss model jointly with the graph structure.
    \item We provide a comprehensive feasibility analysis of the integrated system, demonstrating that despite the inherent localization errors of RSSI (~150-200 m), successful point-to-point navigation and wide-area search missions are achievable through geometric stability and strategic flight path planning.
    \end{enumerate}

Our work distinguishes itself from prior research by integrating infrastructure self-organization and robust client navigation into a unified, resilient system for GNSS-denied environments.
Existing approaches typically address these challenges in isolation: drone localization studies often assume a pre-existing, high-precision anchor network, while sensor localization methods frequently rely on high-density deployments or labor-intensive fingerprinting.
In contrast, we propose an integrated architecture that establishes a navigation capability from scratch.
For the infrastructure layer, we introduce a physically-informed graph learning model that, unlike standard black-box neural networks, embeds radio propagation physics to enable sparse beacon networks to self-localize in complex terrain without GNSS.
Simultaneously, for the mobile client, we diverge from hardware-intensive timing or angle-based techniques by developing a robust multilateration algorithm that leverages altitude constraints and Huber loss to mitigate the inherent volatility of low-power signals.
This combined approach allows us to eliminate the reliance on external positioning systems at both the infrastructure and client levels, offering a standalone solution for unknown environments.

The remainder of this thesis is structured as follows.
We first review related work and detail the system architecture.
Next, we present our robust client localization method, validated through field experiments, followed by the introduction of the physically-informed GCN for infrastructure self-localization.
Finally, we evaluate the integrated system's feasibility and conclude the thesis.

\subsection*{2. Related Work}

\paragraph{Wireless Positioning Fundamentals:}
While Time-Difference-of-Arrival (TDoA) and Angle-of-Arrival (AoA) methods offer higher accuracy than Received Signal Strength Indicator (RSSI), they impose strict hardware constraints.
TDoA typically requires nanosecond-level time synchronization (often derived from GPS), which contradicts the goal of a GNSS-independent backup system.
AoA requires complex antenna arrays.
We therefore focus on RSSI-based methods, which are compatible with standard, low-cost LoRa modules.

\paragraph{Client-Side: Drone Localization Approaches (Phase 1):}
Standard drone navigation relies on GNSS or visual odometry (SLAM).
While SLAM is effective, it is computationally heavy and fails in feature-poor environments (e.g., deserts, oceans).
In the RF domain, existing works typically target high-density indoor environments (e.g., WiFi or Bluetooth grids) or rely on hardware-intensive TDoA/AoA systems.
These approaches are ill-suited for sparse, large-scale outdoor deployments where hardware simplicity is paramount.
Our approach differentiates itself by achieving robustness through algorithmic enhancements (altitude constraints and Huber loss) rather than specialized hardware or high beacon density, enabling operation with simple, single-antenna beacons.

\paragraph{Infrastructure-Side: Network Self-Localization (Phase 2):}
For self-localizing the beacon network, standard multilateration is insufficient as it cannot localize nodes lacking direct connection to at least three anchors.
Its extension, Iterative Multilateration, attempts to bridge this gap by utilizing unknown-to-unknown connections; however, it suffers significantly from catastrophic error propagation in sparse, non-line-of-sight (NLOS) environments.
Fingerprinting methods offer high accuracy but require labor-intensive site surveys, making them unsuitable for rapid deployment in unknown disaster zones.
While Graph Neural Networks (GNNs) have been applied to sensor networks, prior studies often treat the network as a ``black box'' or focus on dense, indoor environments.
Our work is distinct in its application of a \textit{Physics-Aware} GCN specifically designed for sparse, outdoor LPWA networks, where we explicitly embed path-loss physics into the learning process.

\subsection*{3. System Architecture}
To clarify the operational concept, we define the system architecture into two distinct components: the \textbf{Infrastructure} (Beacons) and the \textbf{Client} (Drone).
\begin{itemize}
    \item \textbf{Infrastructure (Beacons):} The ground segment consists of low-cost LoRa beacons. In scenarios where beacon positions are unknown (e.g., aerial deployment), the system initiates a \textbf{Self-Organization Phase}.
    This phase utilizes a Physics-Aware GCN model, pre-trained on synthetic data derived from digital elevation models to capture site-specific propagation characteristics.
    Beacons measure peer-to-peer RSSI, and this data is aggregated at a central hub where the model estimates their positions.
    The hub then transmits the estimated coordinates back to each beacon.
    Once localized (or if positions were already known), the beacons enter a power-saving state, activating \textbf{Broadcast Mode} only upon receiving a wake-up signal.
    \item \textbf{Client (Drone):} The drone primarily operates in a passive, receive-only mode.
    It listens to the global broadcasts from the localized beacons.
    Using the received RSSI and the broadcasted coordinates, the drone performs onboard multilateration to estimate its own position.
    To conserve infrastructure power, the drone transmits a wake-up signal only when broadcast signals are unavailable, maintaining system scalability.
\end{itemize}

\subsection*{4. Phase 1: Robust Drone Localization}
This phase focuses on the ``client side'' localization problem: estimating the drone's position given a set of beacons with known locations.
The primary technical challenge is the high variance of RSSI data caused by multipath propagation and shadowing.

To mitigate this, we enhance the standard multilateration algorithm with two key modifications:
\begin{enumerate}
    \item \textbf{Altitude Decoupling:} Leveraging the availability of reliable barometers or laser altimeters on modern UAVs, we decouple the vertical (Z-axis) estimation from the horizontal (X/Y-axis) positioning.
    This reduces the dimensionality of the optimization problem and prevents vertical errors from propagating into the horizontal plane.
    \item \textbf{Huber Loss Integration:} We replace the standard Least Squares (L2) objective function with the Huber loss function.
    Unlike Least Squares, which penalizes large errors quadratically and is thus highly sensitive to outliers, Huber loss applies a linear penalty to errors exceeding a threshold.
    This ensures that sporadic, extreme RSSI fluctuations do not disproportionately skew the estimated position.
\end{enumerate}

We validate this approach through simulations and real-world field experiments:
\begin{itemize}
    \item \textbf{Simulation Results:} In static hover tests, the proposed method achieves a median accuracy of \textbf{4.5 meters}, a significant improvement over the 29.7 meters of the standard method.
    In dynamic flight scenarios (10 m/s), while the error increases to \textbf{62.9 meters}, the method reduces positioning error by approximately \textbf{25\%} compared to the standard approach (84.1 meters).
    \item \textbf{Experimental Validation:} Field tests conducted in Italy using a commercial drone and four low-cost LoRa beacons confirm the simulation findings.
    The altitude-constrained approach reduces position variance by approximately \textbf{50\%} compared to unconstrained 3D estimation.
    Additionally, the experiments reveal a practical hardware limitation: at higher altitudes (15 meters), the antenna radiation pattern causes signal attenuation directly above the beacon (the ``cone of silence''), necessitating altitude-aware path-loss modeling or alternative antenna configurations in future iterations.
\end{itemize}

\subsection*{5. Phase 2: Infrastructure Self-Localization}
This phase addresses the ``infrastructure side'' problem: the self-localization of the beacon network in a GNSS-denied environment where manual surveying is impossible.
This scenario emulates a rapid deployment where a subset of nodes serve as anchors with known positions (e.g., derived from temporal windows of GNSS availability or established at the spatial perimeter of the denied zone), while the remaining beacons must determine their coordinates solely through peer-to-peer RSSI measurements.

To solve this ill-posed inverse problem, we employ a \textbf{Graph Convolutional Network (GCN)} pre-trained on synthetic terrain data.
Unlike standard data-driven models that treat the network as a black box, our architecture is ``physics-aware.'' It explicitly learns the parameters of a log-distance path-loss model ($n$ and $o$) jointly with the graph structure, allowing the network to adapt to the specific radio propagation characteristics of the local terrain.

We evaluate this method using a high-fidelity, terrain-aware simulation of a 4 km $\times$ 4 km mountainous region with a 25\% anchor allocation (16 anchors out of 64 nodes), employing the Longley-Rice propagation model to generate realistic signal data.
\begin{itemize}
    \item \textbf{Results:} In this challenging sparse network environment (4 beacons/km\textsuperscript{2}), standard iterative multilateration fails completely, yielding a mean error of over \textbf{1900 meters}.
    A standard GCN performs better but still suffers a high error of \textbf{750 meters}.
    The proposed Physics-Aware GCN achieves a mean error of \textbf{206.74 meters}, representing a \textbf{90\% improvement} over the baseline methods.
    \item \textbf{Implication:} While this accuracy represents a significant advancement for RSSI-only self-localization in complex terrain, the remaining error (~200 meters) poses a critical question for the final integrated system: is this level of accuracy sufficient for practical drone operations?
\end{itemize}

\subsection*{6. Integrated Feasibility and Deployment Strategy}
The final phase integrates the results of Phase 1 and Phase 2.
We analyze the full deployment lifecycle, from air-dropping beacons to the final search mission.
Specifically, we simulate a drone navigating using beacon positions estimated by the GCN.

\paragraph{Sensitivity Analysis:}
First, we quantify how infrastructure errors propagate to the drone.
Our simulation shows that adding Gaussian noise ($\sigma=150$ m) to beacon locations degrades the drone's localization error from a baseline of 116 m to 138 m.
This linear degradation confirms that the system is robust; errors in the infrastructure do not cause catastrophic failure on the client side.

\paragraph{Navigation Feasibility:}
We then conduct a closed-loop flight simulation using these noisy estimates.
This analysis yields three critical insights into operational feasibility:

\begin{enumerate}
    \item \textbf{Geometric Stability:} During the long-distance cruise phase ($>$1 km to target), a position error of 150 meters results in a heading error of less than \textbf{10 degrees}.
    This geometric dilution effect ensures that the drone maintains a stable, effectively straight flight path toward the target area, despite significant positional uncertainty.
    \item \textbf{The Turning Point Problem:} As the drone approaches a waypoint ($<$200 meters), the distance-to-error ratio decreases, causing the heading error to spike and resulting in erratic flight behavior.
    We address this by implementing a \textbf{200-meter arrival threshold}, prioritizing mission continuity over precise waypoint arrival.
    \item \textbf{Area Sweep Feasibility:} For search and rescue operations, we demonstrate that by planning flight paths with \textbf{50\% overlap} (e.g., 100-meter spacing for a 200-meter sensor swath), the system facilitates \textbf{effective ground coverage}.
    This strategy effectively trades flight efficiency for reliability, ensuring mission success even with lateral position errors of up to 50 meters.
\end{enumerate}

\subsection*{7. Conclusion}
This thesis demonstrates that while LoRa technology cannot replace GNSS for precision applications, it serves as an effective, survivable backup for emergency navigation.
By coupling a self-organizing ground network (Phase 2) with robust drone navigation algorithms (Phase 1), we establish a functional navigation infrastructure in environments with limited prior infrastructure.
This technology provides a critical capability for defense and disaster response, ensuring that autonomous missions continue even when primary navigation systems fail.

\end{document}
