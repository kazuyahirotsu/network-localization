\documentclass[master,final,11pt]{iscs-thesis}

\usepackage[dvipdfmx]{graphicx}
\usepackage{url}
\usepackage{algorithm}
\usepackage{algorithmic}
%\usepackage{algpseudocode}
\usepackage{tabularx}
\usepackage{booktabs}
\usepackage{multirow}
\usepackage{longtable}
\usepackage{array}
% 論文の種類とフォントサイズをオプションに
%-------------------
\etitle{Anomaly Detection and Congestion Avoidance for Large Scale 5G-IIoT using Unsupervised Online Machine Learning}
\jtitle{教師なしオンライン機械学習による大規模5G-IIoT\\トラフィック異常検出・輻輳回避手法}
%
\eauthor{Yuxuan Shi}
\jauthor{師宇軒}
\esupervisor{Akihiro Nakao}
\jsupervisor{中尾彰宏}
\supervisortitle{Professor} % Professor, etc.
\date{February 20, 2025}
%-------------------
\begin{document}
\begin{eabstract}
The large-scale deployment of Internet of Things (IoT) devices and the increasing traffic volume bring congestion-related challenges to IoT systems. A large-scale IoT system, including a swarm of devices, has certain difficulties in managing a large amount of devices and is weak to intrusions and unexpected defects. These challenges are particularly significant in Industrial IoT (IIoT) environments, where congestion caused by attacks or malfunctions threatens reliability greatly, and the reliability problem is often security-critical. Undetected anomalies, which can be a sign of intrusion or defect taking place, can cascade effects and threaten the system's stability. In worst conditions, the failure of safety-critical systems can result in a halt of production processing or even end up causing security incidents.
The academic society is already working on detecting anomalies in IoT environments, especially for intrusion prevention systems. However, existing methods suffer from the following issues: Control plane (C-Plane) key performance indicators (KPI)-based congestion avoidance methods cannot accurately classify devices based solely on C-Plane features. Existing User Plane (U-Plane) empowered anomaly detection methods are mainly based on supervised learning and rely heavily on labeled datasets. To solve the problems, we propose an anomaly detection and congestion solution system based on U-Plane traffic data and Radio access network intelligent controller (RIC). First, we propose an unsupervised online learning-based anomaly traffic detection and congestion avoidance framework, where U-plane data is utilized to provide accurate classification. Second, we propose an anomaly device isolation mechanism from the main network by steering traffic using a near real-time RIC application (xAPP). Finally, we establish the simulation and hardware experimental platform to evaluate the effectiveness of our proposed system. In the experiment, the proposed method achieves a 94\% classification accuracy, while the accuracy of the traditional method is 63\%.
\end{eabstract}
\begin{jabstract}
IoT(Internet of  Things)デバイスの大規模な普及とトラフィック量の増加は、IoTシステムにおいて混雑に関連する重大な課題を引き起こしている。大規模なIoTシステムは、多数のデバイスを効率的に管理することが困難であり、侵入や予期しない欠陥に対する脆弱性を抱えている。この問題は特に産業用IoT(IIoT)環境において顕著であり、攻撃や障害による混雑がシステムの信頼性を大きく損なうリスクを伴う。さらに、未検出の異常は侵入や欠陥の兆候となる可能性があり、それが連鎖的な影響を引き起こしてシステムの安定性を脅かすことがある。最悪の場合、安全性が求められるクリティカルなシステムが停止し、生産ラインの稼働中断や重大な事故に発展する可能性がある。
これらの課題に対応するため、学術界ではIoT環境における異常検出技術、とりわけ侵入防止システムの研究開発に注力している。しかし、既存の手法には以下のような問題が存在する。制御プレーン(C-Plane)の主要パフォーマンス指標(KPI)に基づく混雑回避手法は、C-Planeの特徴量に依存しているため、デバイスを正確に分類する能力に限界がある。また、ユーザープレーン(U-Plane)を活用した既存の異常検出手法は、主に教師あり学習に基づいており、ラベル付きデータセットに対する過度の依存が課題となっている。
本研究では、これらの問題を解決するため、U-Planeトラフィックデータと無線アクセスネットワークインテリジェントコントローラ(RIC)を活用した異常検出および混雑を解消するシステムを提案する。本提案手法では、U-Planeデータに基づき分類精度を向上させる教師なしオンライン学習手法を採用し、異常トラフィックの検出と混雑回避を実現する仕組みを構築する。さらに、リアルタイムRICアプリケーション(xAPP)を用いてトラフィックを制御し、発見された異常デバイスをメインネットワークから隔離するメカニズムを提案する。最後に、提案手法の有効性を評価するため、シミュレーションおよびハードウェア実験環境を構築し、精度を検証する。
実験結果では、提案手法が94%の分類精度を達成し、従来手法の63%を大幅に上回ることが確認されている。 
\end{jabstract}
\maketitle

\begin{acknowledge}
 I want to express my deepest gratitude and appreciation to my advisor, Professor Akihiro Nakao. Professor Akihiro Nakao has been supportive, friendly, and helpful throughout my master's program. His encouragement and understanding helped me overcome difficulties and have led me the way to develop my research topics and ideas. I have always been fascinated by Professor Akihiro Nakao's insights into information science and his ability to identify and solve problems, and his example in time planning has enabled me to learn and apply it well in my research and daily life. Furthermore, his enthusiasm and knowledge in Networking Technologies have sparked my interest in continuing more profound research in this area, shaped my critical thinking, and challenged me to become a higher-level thinker.
    
    Working with Professor Qianqian Pan has been an enlightening and enriching experience. Whenever I needed advice, she was always quick and patient to help. Moreover, she was always polite and willing to share her experiences and views on persuasive research.  
    
    I also want to thank Assistant Professor Kenji Kanai for his help during the experiment phase and helped me polish the concept during the early stages of my research.
    
    I would also like to thank my family and friends for their continuous support and understanding in my pursuit of scientific research. Without their help, I would not have pursued my studies without worry.

    Finally, I would like to thank the official members of my examining committee for their efforts in reviewing my thesis and providing helpful suggestions.
\end{acknowledge}

\frontmatter %% 前付け
\tableofcontents % 目次
\listoffigures % 図目次
\listoftables % 表目次
%\lstlistoflistings % ソースコード目次
%-------------------
\mainmatter %% 本文

\chapter{Introduction}
\label{chap:intro}

\subsubsection{Background}

The Industrial Internet of Things (IIoT) environment is characterized by its vast number of connected devices, including numerous sensors, industrial equipment, monitors, and recording devices. Such application environments impose significant performance demands on Radio Access Networks (RAN). WiFi and Local 5G are often discussed and compared among the potential solutions for providing radio network services. In the following section, we will analyze the performance requirements of RAN in IIoT environments and compare the advantages and disadvantages of Local 5G and its main competitor, wifi, under the use case and conditions of IIoT. This section answers why we should choose Local 5G instead of WiFi as the RAN system for IIoT, based on the results of in-depth research \cite{wen2022private5g}.


\subsubsection{Performance Requirements of RAN in IIoT Environments}

Private 5G networks, as discussed in \cite{wen2022private5g}, possess advanced features that meet the demands of IIoT environments. One of the most critical requirements is ultra-low latency, which enables real-time communication essential for applications that require synchronization, such as robotic control and machine-to-machine interaction. Even slight delays can lead to significant inefficiencies or operational risks in these scenarios. Furthermore, high reliability is obviously an essential requirement, particularly in industrial processes where interruptions are not tolerated, such as safety-critical warning systems. 

Additionally, IIoT environments are characterized by the massive deployment of connected devices, ranging from sensors to cameras and augmented reality tools for digital twins. These devices collectively demand networks capable of supporting high-density connectivity without performance degradation. 

\subsubsection{Comparison Between WiFi and Local 5G}

\begin{table}[ht]
\caption{Comparison between Wi-Fi 6 and Private 5G (From \cite{wen2022private5g})}
\label{tab:compare}
\resizebox{\columnwidth}{!}{%
\begin{tabular}{|l|l|l|}
\hline
                      & Wifi-6               & Private 5G       \\ \hline
Spectrum              & Unlicensed           & Licensed         \\ \hline
Coverage              & Local                & Wide             \\ \hline
Reliability           & Low                  & High             \\ \hline
Mobility \& off-site  & Low                  & High             \\ \hline
Security              & Low                  & High             \\ \hline
Outdoor suitability   & Low                  & High             \\ \hline
Cost                  & Cheap                & High             \\ \hline
Application scenarios & Non-mission-critical & Mission-critical \\ \hline
\end{tabular}%
}
\end{table}

Although WiFi, particularly in its latest version, such as WiFi 6, has achieved improved capacity and data rates compared to formal standards, it faces inherent limitations in meeting the specific requirements of IIoT applications. Operating mainly in unlicensed spectrum, WiFi is susceptible to interference, leading to variability in performance that is unsuitable for mission-critical tasks. Moreover, its reliance on listen-before-talk mechanisms for channel access increases latency in congested environments, making it less reliable for applications demanding near-instantaneous response times. While WiFi excels in cost-effectiveness and ease of deployment, its inability to meet the rigorous demands of industrial settings often surpasses these benefits.

In contrast, Local 5G networks utilize licensed or dedicated spectrum, providing reliable and interference-free operation. This capability is essential for ultra-reliable low-latency communication (URLLC) in IIoT applications. For example, when Local 5G is integrated with Time Sensitive Networking (TSN), it guarantees deterministic communication, which is vital for automation and real-time control systems. Private 5G effectively meets this challenge by adapting numerous device connections while ensuring steady throughput and low latency, even during heavy usage scenarios, including mainly constant or spiking heavy traffic. Security remains a major concern since IIoT applications often manage sensitive data, and detecting intrusions in extensive networks can be challenging. Essentially, Local 5G functions as a private cellular system that can be customized to specific requirements. For instance, advanced security certification can be implemented by organizing UE groups and verifying SIM details to enforce access restrictions. Additionally, Local 5G offers sophisticated features like vertical network slicing, enabling operators to allocate network resources according to particular applications, thus optimizing performance for various use cases. Its scalability and mobility management capabilities also exceed those of WiFi, making it the preferred option for expansive and dynamic IIoT settings.

\begin{figure}[hb]
    \centering
    \includegraphics[width=1.0\linewidth]{Img/related/l5g_spectrum.png}
    \caption{The Deployment and Spectrum Options for Private 5G Networks (From \cite{wen2022private5g})}
    \label{fig:spectrum}
\end{figure}

\subsubsection{Summary}

Given the performance demands of IIoT settings, Local 5G outperforms its competitor WiFi for RAN solutions. The advantages of private local 5G, such as extremely low latency, high reliability, and extensive connectivity, align well with the needs of IIoT environments. Additionally, its improved security features, customization options, and capacity to sustain connections for mobile devices further establish it as a preferable option for contemporary industrial applications.

\section{Problem Statement}

This thesis aims to develop an anomaly detection-based congestion solution scheme for local 5G-based IIoT system. For the task of congestion solution, three main modules should be included in the scheme - data collection, anomaly detection, and RAN management. 

\begin{description}
    \item[Lacking detailed features] The first problem of this research is the difficulty of providing detailed parameters for anomaly detection by the data collection module. As utilized in current methods, C-Plane KPIs \cite{c-plane_parameter} are the conventional choice for parameter collection. However, the problem is the C-Plane KPI only includes indicators such as throughput, connection count, end-to-end latency, and slice-related indicators. These indicators are designed to observe the workload and monitor the performance of the RAN at the network layer \cite{osi_model} and are not designed for anomaly detection tasks. Therefore, only utilizing C-Plane parameters results in a lack of application layer parameters, which are critical to anomaly detection tasks in the IIoT environment. On the other hand, U-Plane traffic is the raw traffic generated from devices, including application layer parameters. U-Plane traffic parameters captured at the data link layer are widely utilized for anomaly detection tasks and have been proven to be effective \cite{chatterjee2022iot} by providing in-detail information from all captured packets. The most popular U-Plane traffic capture tool, Wireshark \cite{wireshark}, provides thousands of available in-depth parameters from the application layer to below. Therefore, this research needs to enable the utilization of U-Plane traffic in the data collection module instead of C-Plane KPIs in existing research. 
    \item[High Training Cost and Inflexible Classification] The second problem of this research is finding a low-cost, flexible anomaly detection method for the proposed scheme. In previous research, supervised machine learning provides high accuracy for anomaly detection tasks over predefined classes \cite{supervised}. However, supervised learning cannot detect undefined anomalies and requires a labeled dataset for every new batch of training data. As a result, a training dataset must be labeled accurately and include all possible types of anomalies. This results in a high training cost and a lack of flexibility, especially for a large-scale IIoT system, which continuously generates a large amount of data flow, making manual labeling work difficult. Unsupervised learning is considered a low-cost approach compared with supervised one. Since unsupervised learning does not rely on labeled datasets to perform classification and anomaly detection, it has lower operational costs than supervised learning per training batch. Also, as the unsupervised approach dramatically cuts the training cost, online learning can be applied to form a flexible solution that adapts to changes in the IIoT system. However, the unsupervised learning approach also requires a more detailed dataset to maintain accuracy, as it relies on clustering for the classification. Also, the online learning approach requires a constant data collection scheme. As a result, the problem for the proposed scheme is to provide a constant flow of detailed data to feed the anomaly detection module.
    \item[Lacking U-Plane-based Autonomous Solution] The research's final problem is establishing an autonomous scheme to perform the congestion solution based on anomaly detection results. Current studies investigate autonomous solutions to mitigate the congestion mainly by utilizing Radio Access Network (RAN) Intelligent Controller (RIC)\cite{RIC}. RIC provides access to RAN's control plane, enabling machine learning-based methods for network management. As defined in the O-RAN architecture, RIC connects with multiple interfaces for C-Plane KPI information collection. However, the current O-RAN architecture does not have an interface for capturing U-Plane traffic. Therefore, a U-Plane parameter-based anomaly detection scheme cannot interconnect with RIC in the current situation. This research has to design an interface-like structure for the data collection module to collect U-Plane parameters.  
\end{description}




\section{Thesis Statement}

To effectively solve the above-mentioned problems in anomaly detection tasks with the existing scheme, this thesis proposes a congestion solution system for large-scale IIoT based on non-supervised clustering device classification. We propose a novel scheme that utilizes U-Plane parameters to enhance the anomaly detection capability using RIC\cite{RIC}. We capture the generated IIoT traffic at U-Plane, extract U-Plane parameters, and train the anomaly detection model by performing unsupervised online learning. Then, we apply the trained model to detect anomalies in the traffic and utilize RIC to steer traffic in case of congestion. The result indicates the proposed method achieves a 94\% classification accuracy, a 31\% advantage compared with the accuracy of the traditional method, which is 63\%. The proposed system shows advantages in the following aspects:

\begin{description}
    \item[Low-cost Training with High Accuracy] In response to the high model training costs of supervised learning, this thesis proposes an anomaly detection scheme based on unsupervised learning. By adopting an unsupervised learning approach, this method completely eliminates the need for manually labeled datasets, thereby reducing training costs. At the same time, the scheme utilizes U-Plane traffic, which contains detailed features, to achieve high-accuracy anomaly detection while keeping training costs low.
    
    \item[Flexible Anomaly Detection] This study introduces online learning to achieve flexible anomaly detection. The proposed method processes U-Plane traffic streams directly captured from the IIoT network by segmenting them into time-based batches and feeding them into the anomaly detection model for online learning. In this way, the anomaly detection model can periodically adapt to changes in the IIoT network, such as adding new devices or changes in device attributes due to testing or rearrangement. Furthermore, since the proposed system uses a clustering-based anomaly detection algorithm, it detects anomalies through changes in cluster membership or as outliers. This approach overcomes the inherent limitation of traditional supervised learning-based systems that can only detect labeled anomaly types, thus enhancing system flexibility.

    \item[Unattended Autonomous Congestion Solution] The proposed system is a fully unattended autonomous system. From collecting U-Plane traffic to training the model and leveraging RIC to steer traffic from detected anomalous devices during congestion to offload the burden from the main gNB, the proposed system achieves fully automated congestion resolution and anomaly detection.
\end{description}






\section{Related Work}
%partly done need review
\subsection{Quality of Service Management Utilizing RIC}
 RAN Intelligent Controller \cite{RIC}, alongside machine-learning-based RIC xAPPs \cite{RICKPI}, has been proven as a resolution for QoS management in 5G systems. The traditional methods utilize C-Plane KPIs\cite{RICKPIFILE} to sense the load status of each base station. RIC evaluates whether the end users are satisfied by sensing KPIs, including throughput and integrated downlink latency. When the QoS cannot fulfill the requirement, RIC can manage the QoS by steering User Equipment (UE) to the optimized gNodeB (gNB).
 However, RIC in existing studies relies solely on C-Plane parameters and cannot process U-Plane parameters. This limits the accuracy of the device classification task, which is vital for anomaly detection.
 This limits the device classification task, which is vital for anomaly detection. Traditional C-Plane-based RIC has limited features, so the accuracy of device classification tasks is not ideal compared to U-Plane-based methods. 
 The proposed method shares the same congestion detection and traffic steering method as the traditional method. Still, it replaces its C-Plane KPIs-based device classification model with our unsupervised online learning model utilizing U-Plane parameters. 

\subsection{Machine Learning Approaches}

For U-Plane parameters, AI-based device classification methods are widely applied in IoT environments\cite{SmartHome}. These methods are trained to adopt multiple-protocol environments and are limited to non-protocol-specified statistical features, such as packet length, IP addresses, and port numbers. 
However, the IIoT environment usually adopts unified data transmission and uses a single protocol for major workloads, making the multi-protocol capability in traditional methods less suitable. Also, the non-protocol-specified features are insufficient to distinguish IIoT devices since the differences are contained in the message payload. Thus, the conventional method is not designed for and cannot achieve peak performance in the IIoT environment.    

The majority of conventional methods are based on supervised learning to achieve accurate classification for predetermined traffic patterns\cite{cook2019anomaly}. However, these methods rely heavily on labeled datasets and lack flexibility against unknown classes. Compared with supervised learning, unsupervised learning does not rely on labeled datasets to operate and is more flexible against unknown targets. Clustering models require large datasets to identify borders between clusters, which is not ideal for classifying small datasets. The large dataset required by unsupervised learning models could be obtained by implementing online learning. 

Online learning is a method for constantly collecting data and reinforcing the classification model to identify new classes. Existing studies based on supervised learning rarely implement online learning due to the cost of labeling work for every update. However, unsupervised learning does not require a labeling process. Thus, unsupervised online learning could be an economical and flexible choice for anomaly detection tasks in large-scale IIoT systems.
\section{Objectives}

The objectives of this thesis include the following:
\begin{itemize}
    \item Design a scheme to provide detailed U-Plane traffic data for the anomaly detection and congestion solution.
    \item Design an online machine learning-based unsupervised clustering algorithm to perform flexible anomaly detection with a low-cost training method that does not need a manually labeled dataset.
    \item Propose an autonomous scheme that utilizes the RIC to perform traffic steering for detected anomaly devices and redirect the devices to a reserved gNB to offload the congestion on the main network.
    \item Setup a testbed based on real-world local 5G network for the simulation of the proposed system, and evaluate the accuracy.
\end{itemize}


\section{Contributions}

Compared to related work, this thesis proposes a U-Plane-based congestion avoidance RIC xAPP scheme for large-scale 5G-IIoT systems. The proposed scheme combines high-accuracy U-Plane-based anomaly detection with intelligent C-Plane network management enabled by RIC. As a result, the proposed method achieves a 94\% classification accuracy, while the accuracy of the traditional method is 63\%. The contributions and advantages of this proposed method are summarized as follows:
\begin{description}
    \item[Provide Detailed Parameters for Anomaly Detection] A RIC-based U-Plane anomaly detection and congestion solution system is proposed, capable of collecting highly detailed raw U-Plane traffic data and isolating anomaly devices. This structure has enabled RIC's RAN management capability by providing high-accuracy anomaly detection results based on U-Plane data.
    \item[Cost-friendly, Flexible Anomaly Detection Scheme] We propose a traffic classification and anomaly detection method based on unsupervised online learning. The proposed method is designed to collect datasets for the training. The unsupervised online learning approach enables flexible classification, which can detect unknown anomalies and adapt to changes like device reorganization.
    \item[Propose Autonomous Solution utilizing RIC] We introduce an autonomous scheme to capture U-Plane traffic and extract the parameters for machine learning-based anomaly detection. Labeling work is not required as we use unsupervised learning, leading to an autonomous data collection module that automatically collects data independently. Then, we utilize RIC to perform the congestion solution to offload the anomaly devices from the main network. The whole control flow is fully autonomous, without the need for man-in-the-circuit instructions.
    \item[Setting Up Testbed for the evaluation] We establish the simulation and experimental platform to evaluate the effectiveness of the proposed method. Therefore, we can evaluate the capability of the proposed method by putting the system into a simulated IIoT-RAN environment, which mimics the real-world IIoT traffic.

\end{description}
The result proves that the proposed method is effective in performing anomaly detection in our simulated IIoT environment, with a significant advancement of 31\% in overall accuracy compared to the traditional method.
\section{Organization}
This thesis proposes an unsupervised machine learning-based adaptive anomaly detection and congestion avoidance for large-scale 5G-Industrial IoT systems.

The thesis chapters are organized as follows.

\textbf{Chapter 1:} First, We introduce the background of the research, which is the need for RAN management methods in the concept of 5G-IIoT systems. Then, we elaborate on the problem statement, thesis statement, and objectives. We then provide an overview of related works about machine learning-based RAN management and its utilization in IoT environments. 

\textbf{Chapter 2:} In this chapter, we provide a more profound introduction to the background of this study: the Industrial IoT (IIoT) environment. Given that most prior research has primarily focused on traditional IoT environments, such as Smart Home IoT, we also provide an overview of the characteristics of conventional IoT environments. Furthermore, we conduct a comparative analysis between traditional IoT and IIoT, highlighting their key differences and discussing how these differences influence the choice of RAN management strategies. 

We also introduce the private local 5G system, which serves as the RAN backbone for Industrial IoT environments. The discussion included a comparison between private local 5G systems and WiFi, highlighting their respective advantages and limitations in IIoT applications. Additionally, we provide an overview of the core components of the private 5G system, including the 5G Core Network (5GC), the Radio Access Network (RAN), and the RAN Intelligent Controller (RIC), detailing their roles and functionalities.

\textbf{Chapter 3:} This chapter includes a survey of related research in the area of anomaly detection for IIoT, and also consists of our previous research, a U-Plane-based two-level approach for the same topic.

\textbf{Chapter 4:} This chapter presents our proposed system design, including the implementation scenario, the structure of the anomaly detection scheme, the machine learning algorithms we select for the proposed method, and the detailed workflow of the scheme.

\textbf{Chapter 5:} This chapter covers the experiment and evaluation of the proposed system. By deploying the proposed system in a simulated Industrial IoT environment built on our laboratory's original 5G network, we evaluate the accuracy of the proposed system.

\textbf{Chapter 6:} Conclusion of this thesis.

\textbf{Chapter 7:} Outlook of the future work.




\chapter{Technical Background}
\label{chap:technical background}

\section{Difference between Industrial and Traditional IoT}
\label{chap:IIoT}

Previous research on topics including anomaly detection and congestion prevention has primarily focused on traditional IoT environments, such as Smart Home IoT \cite{alaa2017review} \cite{bakar2015activity} \cite{ramapatruni2019anomaly}. However, this thesis examines the industrial IoT (IIoT) environment, which differs in many ways from traditional Smart Home IoT \cite{sisinni2018industrial}. This chapter provides an overview of both industrial and traditional IoT environments, highlighting their key differences. It also examines how these differences influence the design and effectiveness of anomaly detection and congestion prevention solutions.

\subsection{Smart Home IoT}

Smart Home IoT, as the name suggests, refers to network solutions deployed in residential or indoor environments to support a wide range of smart home devices \cite{alaa2017review}. Typical smart home devices include cameras, sensors, voice-active assistants like Amazon Alexa, smart TVs, computers, and mobile phones. In principle, any device within a residential system that connects to the home's wireless network (usually Wi-Fi) falls under this category. As a result, the key characteristics of Smart Home IoT are the diversity of device types, significant differences in communication structures, and a relatively small number of devices installed with limited data transmission volumes (compared to Industrial IoT, or IIoT). However, due to the privacy-sensitive nature of residential environments, the demand for anomaly detection in smart home settings is much higher than in IIoT.

Given these characteristics, anomaly detection solutions tailored for Smart Home IoT have their unique features \cite{chatterjee2022iot}. Smart Home IoT does not typically require robust congestion prevention solutions since the number of devices installed is relatively small and high-volume communications rarely occur simultaneously, compared with IIoT. However, the high priority on privacy protection in smart home environments creates a strong demand for both external intrusion detection and internal anomaly detection systems. Consequently, a considerable count of research has focused on anomaly detection in Smart Home IoT \cite{bakar2015activity}. The diverse communication protocols used in smart home environments also present significant challenges for anomaly detection tasks. Unlike IIoT, where detecting anomalies among multiple similar devices is common, anomaly detection in Smart Home IoT is often device-specific. In this context, traditional rule-based firewall architectures usually outperform machine learning approaches in terms of accuracy, as computational resources are relatively sufficient. This is because machine learning methods struggle to deliver high precision in such environments.

\subsection{Industrial IoT}
Industrial IoT (IIoT) refers to IoT network systems deployed in industrial environments such as smart factories. Unlike Smart Home IoT, which involves fewer devices with diverse functionalities, IIoT is designed to support a large number of industrial devices that are varied in multiple types \cite{sisinni2018industrial}. These devices, primarily sensors and controllers, are distributed across factory facilities. In contrast to the diverse communication protocols employed in Smart Home IoT, IIoT environments typically use fewer or even a single protocol for data transmission. This standardization is largely due to the fact that all sensors and controllers are integrated into a centralized control system. While the importance of external intrusion and anomaly detection is equally critical in IIoT and Smart Home IoT environments, the massive number of devices and the inclusion of time-sensitive components—such as controllers requiring millisecond-level synchronization—make IIoT systems more susceptible to congestion. Furthermore, because of the large scale of devices and their sensitivity to interference and latency, Wi-Fi, with its lower interference resistance and limited communication capacity, is rarely used in IIoT environments. Instead, private local 5G networks are often adopted for connectivity.

These unique characteristics mean that congestion prevention and anomaly detection solutions in IIoT environments differ significantly from those developed for Smart Home IoT \cite{ramapatruni2019anomaly}. Solutions for IIoT must be capable of handling the massive communication generated by high-concurrency operations among numerous devices, as this is a fundamental requirement for IIoT systems. Such demands necessitate high processing capabilities and often lead to the adoption of control-plane key performance indicator (C-Plane KPI) analysis rather than user-plane (U-Plane) packet capture for anomaly detection. C-Plane KPI analysis offers considerable advantages in terms of acquisition and processing speed compared to U-Plane traffic analysis. Combined with the RAN Intelligent Controller (RIC), it enables near-real-time control loops, allowing the system to respond swiftly to anomalies—an essential feature in IIoT environments, where minimizing downtime is critical.

Additionally, the large number of identical devices and the uniformity of protocols make machine learning an optimal approach for IIoT control loops. Supervised learning-based methods can quickly and accurately identify labeled anomalies, while unsupervised learning-based methods are highly effective in detecting unknown anomalies. Their flexibility and adaptability allow them to localize anomalous devices in real time, facilitating emergency responses and supporting subsequent manual decision-making.

\subsubsection{MQTT Protocol}
This subsection introduces the MQ Telemetry Transport (MQTT) protocol \cite{soni2017survey}, a lightweight protocol designed to meet the needs of IoT systems. With appropriate client implementations, MQTT enables high concurrency while stressing less on servers compared to conventional protocols \cite{rao2015implementing}\cite{naik2017choice}.

Conventional protocols (e.g., TCP, UDP) were originally designed for internet-based communication. In such environments, factors like transmission efficiency (the ratio of payload to the overall packet size) are not always the highest priority. Additionally, traditional protocols exhibit several characteristics that are not fully suited to Industrial IoT applications. To address these limitations, the MQTT protocol was developed. It is specifically tailored for IoT environments, effectively meeting the transmission demands and communication constraints of IIoT devices. As the comparison between MQTT and other IoT protocols is shown in Fig. \ref{fig:protocol-compare}, the key features of the MQTT protocol are as follows:
\begin{description}
    \item[Lightweight Protocol] In Industrial IoT environments, where a large swarm of devices generates high volumes of traffic, lightweight communication protocols are crucial for alleviating communication pressure. MQTT is based on the conventional TCP protocol but optimizes several unnecessary components in IoT contexts. This streamlining significantly improves transmission efficiency, reducing the overhead associated with communication.
    \item[Centralized Broker-Client Structure] Edge IoT devices often have limited processing capabilities, making the computational burden on clients a significant challenge. To address this issue, MQTT employs a centralized broker-client architecture. In this structure, all communication is initiated by clients and processed centrally by the MQTT broker before being distributed. This design shifts the computational burden to the broker, significantly reducing the processing demands on clients. Consequently, the protocol can operate efficiently on low-performance edge IoT devices, lowering the hardware requirements for connected devices.
    \item[Topic-based Multithread] MQTT utilizes a topic-based multithreading model for communication management. Each topic includes multiple publishers and subscribers. When a publisher pushes information to the broker, the broker automatically broadcasts the message to all subscribers of that topic. Each edge IoT device can execute multiple threads simultaneously, allowing it to listen to one topic while publishing information to another, and vice versa. This multithreading mechanism provides flexibility and scalability for communication in IoT environments.
\end{description}

\begin{figure}
    \centering
    \includegraphics[width=1\linewidth]{Img/related/protocol-cpmpare.png}
    \caption{A comparison between MQTT and Other Lightweight IoT Protocols. from \cite{naik2017choice}}
    \label{fig:protocol-compare}
\end{figure}

\subsection{Summary}

This section provides an overview of the difference between traditional IoT environments (using smart home IoT as a sample) and Industrial IoT. This chapter also compares the anomaly detection methods used in conventional and industrial IoT, based on their respective different environments. The comparison is summarized in the Table. \ref{tab:IoT-compare}. Additionally, this chapter briefly introduces the MQTT protocol, one of the most used lightweight elastic protocols for the IIoT environment.

\begin{table}[h]
\caption{Comparison between Conventional Smart Home and Industrial IoT}
\label{tab:IoT-compare}
\resizebox{\columnwidth}{!}{%
\begin{tabular}{|l|l|l|}
\hline
                              & Conventional IoT & Industrial IoT \\ \hline
Device Quantity               & Low              & High           \\ \hline
Device Class Diversity        & High             & Low            \\ \hline
Device Count in Class         & Low              & High           \\ \hline
Latency Tolerance             & High             & Extremely Low  \\ \hline
Protocol Diversity            & High             & Low (Unified)  \\ \hline
Access Method                 & Mainly Wi-Fi     & Local 5G       \\ \hline
Anomaly Detection Requirement & Strict           & Relatively Low \\ \hline
Managemnt Basis               & Device-based     & Class-based    \\ \hline
\end{tabular}%
}
\end{table}
\section{Local 5G System}
\label{sec:local5g}
\subsubsection{Background}

The Industrial Internet of Things (IIoT) environment is characterized by its vast number of connected devices, including numerous sensors, industrial equipment, monitors, and recording devices. Such application environments impose significant performance demands on Radio Access Networks (RAN). WiFi and Local 5G are often discussed and compared among the potential solutions for providing radio network services. In the following section, we will analyze the performance requirements of RAN in IIoT environments and compare the advantages and disadvantages of Local 5G and its main competitor, wifi, under the use case and conditions of IIoT. This section answers why we should choose Local 5G instead of WiFi as the RAN system for IIoT, based on the results of in-depth research \cite{wen2022private5g}.


\subsubsection{Performance Requirements of RAN in IIoT Environments}

Private 5G networks, as discussed in \cite{wen2022private5g}, possess advanced features that meet the demands of IIoT environments. One of the most critical requirements is ultra-low latency, which enables real-time communication essential for applications that require synchronization, such as robotic control and machine-to-machine interaction. Even slight delays can lead to significant inefficiencies or operational risks in these scenarios. Furthermore, high reliability is obviously an essential requirement, particularly in industrial processes where interruptions are not tolerated, such as safety-critical warning systems. 

Additionally, IIoT environments are characterized by the massive deployment of connected devices, ranging from sensors to cameras and augmented reality tools for digital twins. These devices collectively demand networks capable of supporting high-density connectivity without performance degradation. 

\subsubsection{Comparison Between WiFi and Local 5G}

\begin{table}[ht]
\caption{Comparison between Wi-Fi 6 and Private 5G (From \cite{wen2022private5g})}
\label{tab:compare}
\resizebox{\columnwidth}{!}{%
\begin{tabular}{|l|l|l|}
\hline
                      & Wifi-6               & Private 5G       \\ \hline
Spectrum              & Unlicensed           & Licensed         \\ \hline
Coverage              & Local                & Wide             \\ \hline
Reliability           & Low                  & High             \\ \hline
Mobility \& off-site  & Low                  & High             \\ \hline
Security              & Low                  & High             \\ \hline
Outdoor suitability   & Low                  & High             \\ \hline
Cost                  & Cheap                & High             \\ \hline
Application scenarios & Non-mission-critical & Mission-critical \\ \hline
\end{tabular}%
}
\end{table}

Although WiFi, particularly in its latest version, such as WiFi 6, has achieved improved capacity and data rates compared to formal standards, it faces inherent limitations in meeting the specific requirements of IIoT applications. Operating mainly in unlicensed spectrum, WiFi is susceptible to interference, leading to variability in performance that is unsuitable for mission-critical tasks. Moreover, its reliance on listen-before-talk mechanisms for channel access increases latency in congested environments, making it less reliable for applications demanding near-instantaneous response times. While WiFi excels in cost-effectiveness and ease of deployment, its inability to meet the rigorous demands of industrial settings often surpasses these benefits.

In contrast, Local 5G networks utilize licensed or dedicated spectrum, providing reliable and interference-free operation. This capability is essential for ultra-reliable low-latency communication (URLLC) in IIoT applications. For example, when Local 5G is integrated with Time Sensitive Networking (TSN), it guarantees deterministic communication, which is vital for automation and real-time control systems. Private 5G effectively meets this challenge by adapting numerous device connections while ensuring steady throughput and low latency, even during heavy usage scenarios, including mainly constant or spiking heavy traffic. Security remains a major concern since IIoT applications often manage sensitive data, and detecting intrusions in extensive networks can be challenging. Essentially, Local 5G functions as a private cellular system that can be customized to specific requirements. For instance, advanced security certification can be implemented by organizing UE groups and verifying SIM details to enforce access restrictions. Additionally, Local 5G offers sophisticated features like vertical network slicing, enabling operators to allocate network resources according to particular applications, thus optimizing performance for various use cases. Its scalability and mobility management capabilities also exceed those of WiFi, making it the preferred option for expansive and dynamic IIoT settings.

\begin{figure}[hb]
    \centering
    \includegraphics[width=1.0\linewidth]{Img/related/l5g_spectrum.png}
    \caption{The Deployment and Spectrum Options for Private 5G Networks (From \cite{wen2022private5g})}
    \label{fig:spectrum}
\end{figure}

\subsubsection{Summary}

Given the performance demands of IIoT settings, Local 5G outperforms its competitor WiFi for RAN solutions. The advantages of private local 5G, such as extremely low latency, high reliability, and extensive connectivity, align well with the needs of IIoT environments. Additionally, its improved security features, customization options, and capacity to sustain connections for mobile devices further establish it as a preferable option for contemporary industrial applications.

\subsection{Mobile Cellular Network Architecture Overview}

Mobile cellular networks form the backbone of modern wireless communication systems, enabling connectivity for a wide range of devices, from traditional mobile phones to advanced Internet of Things (IoT) devices and industrial machinery. In the context of IIoT environments, private local 5G networks offer a novel approach to addressing the performance demands of mission-critical applications. This section provides an overview of the architectural components and functionalities of private local 5G networks, emphasizing their suitability for industrial use cases.

Private local 5G networks are structured around two primary subsystems: the Radio Access Network (RAN) and the Core Network. These components are paired together to deliver the ultra-reliable, low-latency, and secure connectivity required for IIoT environments. As illustrated in Fig. \ref{fig:network_architecture}, the RAN includes multiple gNodeB (next generation Node B), which are the base stations, and the User Equipment (UE), which are the devices connected to the network. The 5G Core provides IP connectivity to the RAN and authenticates UEs, as well as overwatch the whole network to ensure the required QoS is satisfied. 

The following sections will provide an in-depth description of the two parts based on the research of Professor Larry Peterson et al\cite{systemsapproach}. 

\begin{figure}[h]
    \centering
    \includegraphics[width=0.8\textwidth]{Img/related/network_architecture.png}
    \caption{Architecture of private local 5G networks illustrating the RAN and Core Network components. Adapted from \cite{systemsapproach}.}
    \label{fig:network_architecture}
\end{figure}



\subsection{5G Core Network}

The 5G Core Network (5GC) is the backbone of the 5G system, enabling advanced features such as network slicing, enhanced security, and seamless mobility. Designed with a service-based architecture, the 5GC provides modularity and scalability to support diverse 5G use cases, ranging from enhanced mobile broadband (eMBB) to ultra-reliable low-latency communication (URLLC) and massive machine-type communication (mMTC). 

This section provides a brief introduction to the 5G core network structure, functions, and components. It also introduces the two split planes in the 5G network: the control plane and the user plane.

\subsubsection{5GC Functions and Components}

\begin{figure}
    \centering
    \includegraphics[width=\linewidth]{Img/related/5gc.png}
    \caption{A structural overview of 5G Mobile Core, represented as a collection of microservices. From \cite{systemsapproach}}
    \label{fig:5gc}
\end{figure}

5G Core is built around several micro-services for its functions, known as the 3GPP Service Based Architecture, illustrated in Fig. \ref{fig:5gc}. As the 5G split control plane and user plane, the components for network and access control tasks and the User Plane Function (UPF) are viewed as two separate planes, which we will make a closer introduction in the following subsections. Internal 5GC control plane services are not our main topic, and we will focus on the Access and Mobility Management Function (AMF), Session Management Function (SMF) and User Plane Function (UPF) in the following subsections. Below is the chart for reference of the remaining control plane microservices:

\begin{itemize}
    \item AUSF (Authentication Server Function): Authenticates UEs.
    \item UDM (Unified Data Management): Manages user identity, including the generation of authentication credentials and access authorization.
    \item UDR (Unified Data Repository): Manages user static subscriber-related information.
    \item UDSF (Unstructured Data Storage Function): Used to store unstructured data, and so is similar to a key-value store.
    \item NEF (Network Exposure Function): Exposes select capabilities to third-party services, and so is similar to an API Server.
    \item NRF (Network Repository Function): Used to discover available services (network functions), and so is similar to a Discovery Service.
    \item PCF (Policy Control Function): Manages the policy rules for the rest of the Mobile Core CP.
    \item NSSF (Network Slice Selection Function): Manages how network slices are selected to serve a given UE.
\end{itemize}


\subsubsection{Control Plane KPIs and Management}

\begin{figure}
    \centering
    \includegraphics[width=1\linewidth]{Img/related/cplane.png}
    \caption{SD-Core implementation of the Mobile Core Control Plane, including support for Standalone (SA) deployment of both 4G and 5G. from \cite{systemsapproach}}
    \label{fig:c-plane}
\end{figure}

The control plane (C-Plane), as shown in Fig. \ref{fig:c-plane} in the 5GC is responsible for signaling, session management, and mobility management. Key responsibilities include:

\begin{itemize}
    \item \textbf{Signaling and Authentication:} Ensuring secure and efficient communication between UEs and the network.
    \item \textbf{Policy and QoS Management:} Enforcing network policies and maintaining quality of service for various applications.
    \item Session Management: Establishing and maintaining sessions for data transfer.
    \item Mobility Management: Seamlessly handling UE mobility across different cells and access networks.
\end{itemize}

The control plane's efficiency and robustness are critical to the reliability and performance of 5G networks. According to 3GPP definitions \cite{RICKPIFILE}, several KPIs are selected to observe the C-Plane performance and are used as input in network management. Table. \ref{table:kpi-summary} is a summary of KPIs defined by the standard, data types, and short descriptions. Existing research \cite{c-plane_parameter} utilizes the RAN Intelligent Controller (RIC) in network management based on C-Plane KPIs. 

\begin{table}[h]
\centering
\renewcommand{\arraystretch}{1.5} % Optional: Adjust row height for better readability
\setlength{\tabcolsep}{4pt}       % Optional: Adjust column spacing
\caption{C-Plane Key Performance Indicators (KPIs) Summary}
\label{table:kpi-summary}
\resizebox{\textwidth}{!}{%
\begin{tabular}{|p{5cm}|p{3cm}|p{8cm}|}
\hline
\textbf{KPI Name} & \textbf{Type} & \textbf{Description} \\ \hline
Mean Registered Subscribers & Cumulative (Integer) & Mean number of subscribers registered to a network slice. \\ \hline
Registration Success Rate & Ratio & Percentage of successful registration procedures. \\ \hline
DRB Accessibility & Ratio & Success rate of Data Radio Bearer (DRB) setup. \\ \hline
PDU Session Establishment Success Rate & Ratio & Percentage of successful PDU session establishment requests. \\ \hline
End-to-End Latency & Mean (ms) & Average latency experienced in the network slice or RAN. \\ \hline
Upstream Throughput & Cumulative (kbit/s) & Total data throughput in the upstream direction. \\ \hline
Downstream Throughput & Cumulative (kbit/s) & Total data throughput in the downstream direction. \\ \hline
QoS Flow Retainability & Ratio & Retainability rate of quality of service flows. \\ \hline
Energy Efficiency & Mean & Data energy efficiency in NG-RAN. \\ \hline
Mobility Handover Success Rate & Ratio & Percentage of successful inter-gNB handovers. \\ \hline
\end{tabular}%
}
\end{table}

Each KPI is described with detailed formulas, measurement levels, and target applicability (e.g., NetworkSlice, SubNetwork, NRCellDU). The detailed specifications for each KPI can be referenced in the document \cite{RICKPIFILE}.

\subsubsection{User Plane}

The user plane (U-Plane) in the 5GC is focused on data forwarding and traffic management. Its primary roles include:

\begin{itemize}
    \item \textbf{Data Forwarding:} Routing user data packets between UEs and external networks.
    \item \textbf{Traffic Steering:} Implementing policies to direct traffic flows according to network slicing and QoS requirements.
    \item \textbf{Low-Latency Processing:} Ensuring minimal delay for latency-sensitive applications like autonomous driving and remote surgery.
\end{itemize}

By separating the control and user planes, the 5GC achieves flexibility and efficiency, supporting diverse 5G use cases while optimizing resource utilization.

As the user plane manages the traffic between UEs and external networks, the packets can be captured at the user plane for deeper and in-detail network analysis. Wireshark \cite{wireshark} is one of the tools that is capable of capturing user plane traffic for this task.
\subsection{Radio Access Network}

The Radio Access Network (RAN) is a fundamental component of modern cellular architecture, serving as the critical link between user devices and the core network. Its primary role is to manage radio resources, facilitating seamless transmission and reception of data packets across the network infrastructure. With the emergence of 5G, RAN design has evolved significantly, placing greater emphasis on flexibility, scalability, and software-defined solutions to meet the demands of next-generation connectivity.

\subsubsection{Packet Processing Pipeline}

The packet processing pipeline is at the heart of RAN operations, ensuring efficient data handling between user equipment (UE) and the network. This pipeline consists of several interconnected stages. Initially, the pipeline receives data packets from user devices or upstream network elements. Once acquired, the packets undergo classification to determine their type and priority, ensuring that Quality of Service (QoS) requirements are met. Following classification, resources are dynamically allocated, and scheduling is performed to optimize throughput and minimize latency. Finally, packets are forwarded to their intended destination with minimal delay, completing the transmission process. In 5G networks, where ultra-reliable and low-latency communications (URLLC) are critical, the pipeline must maintain high efficiency and low latency to meet stringent performance standards.

\subsubsection{Software-Defined RAN}

The evolution of RAN has been shaped by the introduction of software-defined principles, giving rise to Software-Defined Radio Access Networks (SD-RAN). By decoupling hardware and software components, SD-RAN introduces significant flexibility in network management and optimization. A centralized controller manages distributed radio units, enhancing coordination and resource utilization. This approach allows operators to dynamically adapt network behavior through software updates, enabling the rapid deployment of new features and services. Additionally, SD-RAN reduces reliance on specialized hardware, lowering capital expenditure and supporting scalable network deployments. These characteristics make SD-RAN a cornerstone of 5G, enabling innovations such as network slicing and enhanced mobile broadband.

\subsubsection{O-RAN Architecture and Interfaces}

The Open Radio Access Network (O-RAN) architecture is designed to bring openness, flexibility, and intelligence to traditional RAN systems. By disaggregating hardware and software components, O-RAN promotes multi-vendor interoperability and supports programmable network features. Central to this architecture are three key components: the Central Unit (CU), which handles non-real-time functions like mobility management and signaling; the Distributed Unit (DU), responsible for real-time processing tasks such as scheduling and radio link control; and the Radio Unit (RU), which interfaces with the physical layer to transmit and receive radio signals. Together, these components enable a highly modular and efficient architecture.

\begin{figure}
    \centering
    \includegraphics[width=1\linewidth]{Img/related/o-ran-architecture.png}
    \caption{O-RAN Architecture and Interfaces}
    \label{fig:o-ran-architecture}
\end{figure}

O-RAN employs a service-based architecture, where standardized interfaces and APIs facilitate communication between network functions. The Open Fronthaul Interface supports communication between the RU and DU, replacing proprietary solutions with open standards. The E2 Interface connects the Near-Real-Time RAN Intelligent Controller (Near-RT RIC) with the CU/DU, enabling dynamic resource management. Meanwhile, the A1 Interface links the Non-Real-Time RIC with the Near-RT RIC to facilitate policy updates and AI/ML model integration. Finally, the O1 Interface supports management and orchestration, connecting the Service Management and Orchestration (SMO) platform to RAN components.

The O-RAN architecture offers several key benefits. It fosters interoperability, enabling a multi-vendor ecosystem through standardized interfaces. Its modular design allows for flexible deployments and upgrades without disrupting the network. By incorporating AI/ML-driven control and automation through RICs, it introduces intelligence to network operations. Additionally, the architecture reduces dependency on proprietary hardware, supporting cost-effective scaling. These features position O-RAN as a critical enabler for agile and efficient networks that meet the diverse demands of 5G and beyond.

\subsubsection{Summary}

The Radio Access Network remains a cornerstone of cellular communication, continuously adapting to the requirements of modern connectivity. Advances in packet processing pipelines and the adoption of software-defined architectures have made RAN more flexible and efficient, capable of supporting the diverse use cases enabled by 5G. These developments ensure that RAN continues to play an integral role in the telecommunication ecosystem, paving the way for future innovations.
\subsection{RAN Intelligent Control}

\subsubsection{RAN Intelligent Controller}

The RAN Intelligent Controller (RIC) is a central element in modern Radio Access Networks (RANs), particularly within the Open RAN (O-RAN) architecture. It introduces programmability and intelligence into RAN operations, enabling real-time and non-real-time optimization of network resources. Designed to enhance the efficiency, flexibility, and intelligence of the Radio Access Network (RAN), the standout features are programmability, AI/ML Integration, and its base on open interfaces. The programmability of RIC supports the development and deployment of custom applications for RAN management, known as xApps and rApps. 

\subsubsection{near-realtime xAPPs and non-realtime rAPPs}

xAPPs are for near-realtime management, which means the control loop is finished in no longer than 10miliseconds to 1 second. rAPPs are non-realtime, which means the response time is normally longer than 1 second. xAPPs achieve rapid response management by feeding on the KPIs provided by E2, while rAPPs are capable of interacting with more data as input from other sources. These applications enable dynamic network optimization without requiring hardware modifications, offering operators outstanding flexibility in adapting to evolving network demands. The difference between non-realtime and near-realtime control loops can be summarized as the follows:

\begin{itemize}
    \item \textbf{Non-Real-Time Control Loop:} Managed by the Non-RT RIC, this loop focuses on long-term network optimization, leveraging advanced analytics and AI/ML models. Key functions include policy generation, machine learning model training, and high-level orchestration. The Non-RT RIC communicates its outputs to the Near-RT RIC via the A1 interface, enabling predictive and policy-driven control.
    \item \textbf{Near-Real-Time Control Loop:} Operated by the Near-RT RIC, this loop addresses latency-sensitive tasks with timescales ranging from milliseconds to seconds. It handles resource allocation, interference management, and dynamic adjustments in response to real-time network conditions. Communication between the Near-RT RIC and RAN components occurs via the E2 interface.
\end{itemize}

RIC xAPP and rAPPs are bridges between RAN management and AI/ML models and provide a way to utilize advanced machine-learning models to analyze vast volumes of network data. These models enable predictive and adaptive adjustments to network operations, improving performance and resource utilization. According to the O-RAN definition, RIC leverages open interfaces such as A1 and E2, enabling seamless communication with RAN components and fostering interoperability across multiple vendors. This open framework ensures that operators can create diverse, multi-vendor network ecosystems while optimizing their networks in near-real-time, aligning with the broader goals of 5G technology.

\subsubsection{Control loops of RIC}

As the bridge between RAN (Radio Access Network) management and machine learning models, RIC (RAN Intelligent Controller) applications rely heavily on their control loop, which is the most important workflow. The general workflow for a single operation performed by the RIC can be summarized as follows:

\begin{description}
    \item [Data Collection] The RIC gathers telemetry data from various RAN elements, including user behavior, network performance metrics, and environmental factors.
    \item [Analysis and Decision-Making] AI/ML models and analytics engines process the collected data to identify optimization opportunities or potential issues.
    \item [Action Execution] The RIC applies control decisions to the RAN through interfaces like E2 and A1, ensuring real-time or policy-driven changes are implemented effectively.
    \item [Feedback Loop] Continuous monitoring evaluates the effectiveness of actions taken, feeding insights back into the decision-making process for iterative improvements.
\end{description}

Based on the above workflow, RIC and machine learning models can be utilized for RAN management to achieve various specific control objectives. Previous research \cite{sun2023intelligent} has summarized the existing methods for achieving these goals through RIC, which can be briefly categorized as follows:

\begin{description}
    \item [Traffic Management] Dynamically allocates resources to manage varying traffic loads and ensure QoS requirements are met.
    \item [Energy Optimization] Adjusts power levels and resource usage to minimize energy consumption without compromising service quality.
    \item [Interference Mitigation] Proactively addresses interference issues to maintain signal quality and network stability.
    \item [Quality of Service (QoS) Optimization:] Ensures consistent user experience by prioritizing critical traffic and allocating resources efficiently.
\end{description}

Control loops are vital for transforming RANs into intelligent and autonomous systems. By integrating real-time adaptability with long-term strategic insights, they enable 5G networks to meet the diverse and evolving demands of modern connectivity, ensuring efficiency, reliability, and user satisfaction.






\section{Summary}

In this chapter, we provide a brief overview of supervised learning and unsupervised learning, the two approaches in machine learning. We compare their training methods, dataset requirements, and other key aspects. We then examine their performance in terms of accuracy, adaptability to new classes, and training cost. Based on this comparison, we arrive at the following conclusions:

While supervised learning demonstrates superior accuracy in static environments (where no devices are added, removed, or modified) and offers the advantage of producing results in readable natural language tags, its limitations become apparent in Industrial IoT (IIoT) environments. Given that IIoT environments are dynamic—with evolving device configurations and constantly changing attack methods—unsupervised learning approaches provide outstanding benefits in terms of reduced training costs and improved adaptability to unknown anomalies and class shifts.

Additionally, we discuss online learning as a method that significantly improves the ability to address the inextricable interlinkage between the multiple phases of model training, testing, and verification. In today’s rapidly evolving environments, where these phases are tightly integrated, online learning offers a practical and effective way to adapt to new threats and operational changes in real-time.

\chapter{Related Work}
\label{chap: related work}

This chapter summarized the existing research on anomaly detection in IIoT environments. This chapter also includes our previous research: a U-Plane-based two-level anomaly detection scheme for large-scale 5G-IIoT: an open-source approach for Aether Onramp \cite{shi2024uplane}.

\section{A survey of related research}
\subsection{Industrial IoT Anomaly Detection: Overview}
The Industrial Internet of Things (IIoT) integrates sensors, networks, and data analytics to enable connectivity and intelligent management of devices, systems, and processes in industrial environments \cite{lee2014industrial}. However, the complexity, diversity, and real-time nature of IIoT data pose challenges for anomaly detection \cite{chandola2009anomaly}. Anomaly detection involves identifying events or data points that deviate from expected behavior, playing a crucial role in fault prediction, efficiency enhancement, and safety assurance. Recent research has explored anomaly detection algorithms based on statistical methods, machine learning, and deep learning, applying them to various industrial scenarios such as predictive maintenance, cybersecurity, and quality control \cite{hodge2004survey}.

\subsection{Categorization of Anomalies in Industrial IoT}
Before performing anomaly detection, defining what constitutes an anomaly is essential. Based on their characteristics, anomalies can generally be classified into the following categories:

\begin{description}
    \item[Point Anomalies] a single data point's behavior deviates from the normal pattern, often caused by sensor errors or short-term events \cite{hawkins1980identification}.
    \item[Contextual Anomalies] Data points that are counted as anomaly in specific contexts, such as a sudden increase in nighttime energy consumption \cite{chandola2009anomaly}.
    \item[Collective Anomalies] Anomaly behavior observed in a group of data points, such as sustained vibrations exceeding normal thresholds \cite{breunig2000lof}.
\end{description}

Identifying anomalies requires determining thresholds, specifically how far a data point must deviate from the normal cluster to qualify as an anomaly. Anomaly detection relies on modeling normal behavior, and integrating contextual and historical information from time-series data is a key challenge \cite{markou2003novelty}. Due to the difficulty of manually selecting parameters and thresholds in large-scale IIoT environments, machine-learning approaches have been introduced to perform anomaly detection.

\subsection{Categorization of Machine Learning Approaches}
The vast data streams and dynamic traffic patterns in IIoT surpass the adaptability of traditional ruleset-based methods. Moreover, manual supervision for rule-setting and adjustment often leads to unacceptable delays in this situation. As a result, machine learning methods capable of autonomous parameter selection play a pivotal role in IIoT anomaly detection. These methods are categorized as follows:

\begin{description}
    \item[Supervised Learning] Requires labeled training data and uses classifiers (e.g., SVM) to identify anomalies \cite{vapnik1999nature}. Although accurate, this approach depends on sufficient labeled datasets. Without adequate labeled data, it fails to recognize new classes or anomalies, leading to misclassification.
    \item[Unsupervised Learning] Detects anomalies through data distribution or clustering methods, such as the density-based DBSCAN algorithm \cite{ester1996density}. This method is suitable for large-scale unlabeled data but can be sensitive to parameter settings.
    \item[Self-Supervised Learning] Trains using pseudo-labels in an unlabeled environment and has recently demonstrated strong adaptability \cite{chen2020simple}.
    \item[Deep Learning] Techniques such as LSTM, Autoencoders (AE), and Variational Autoencoders (VAE) excel in detecting anomalies in time-series and high-dimensional data \cite{hochreiter1997long, kingma2013auto}.
\end{description}


\subsection{Applications in Industrial IoT}
Anomaly detection in IIoT has broad applications, extending beyond congestion-related anomaly detection discussed in this thesis to monitoring device conditions and providing early warnings across the entire IIoT network.

\begin{description}
    \item[Device Condition Monitoring] Detects equipment faults using vibration signals and temperature data. Typical methods include Mahalanobis distance and wavelet transforms.
    \item[Smart Manufacturing] Optimizes production processes by monitoring sensor data and employing deep learning methods (e.g., CNN) to improve product quality \cite{krizhevsky2012imagenet}.
    \item[Cybersecurity] Identifies malicious activities and data breaches in IoT networks, such as DDoS attack detection \cite{zhang2013privacy}.
    \item[Smart Energy Management] Uses smart meter data to optimize energy distribution and detect energy theft \cite{depuru2011electricity}.
\end{description}

\subsection{Challenges}
Despite significant progress in related research, IIoT anomaly detection still faces major challenges associated with the intrinsic features of IIoT systems, traffic patterns, and the performance constraints of edge devices.

\begin{description}
    \item[Non-stationary Pattern] Algorithms must dynamically adapt to changes in statistical properties over time, such as traffic pattern variations among similar device types \cite{gama2010knowledge}.
    \item[Multimodal Data Fusion] Combining data from different sensors and utilizing their spatiotemporal correlations is essential \cite{baltrusaitis2019multimodal}. As an example for this thesis, integrating C-Plane KPIs with detailed U-Plane parameters is critical for accurate classification.
    \item[Resource Constraints] Lightweight algorithm design is vital for edge devices, which often lack significant computational capacity and rely on servers for analytical tasks \cite{sze2017efficient}.
    \item[Privacy and Security] Anomaly detection methods must preserve user privacy and data integrity. This involves minimizing the decryption of encrypted IIoT traffic and extracting limited session information (e.g., TCP session details) \cite{li2017efficient}.
\end{description}


\subsection{Summary}
This Section reviewed the current state and categorization of IIoT anomaly detection methods and discussed the application scenarios and suitability of different machine learning techniques. While existing technologies have achieved efficient detection in some areas, the complexity of industrial environments demands higher standards for anomaly detection methods. In summary, IIoT anomaly detection methods must enable real-time, autonomous detection in large-scale, resource-constrained environments by combining diverse data sources, minimizing decryption efforts, and intelligently selecting features and thresholds. This necessitates designing a flexible and autonomous anomaly detection framework.

\section{Aether Onramp based Previous Research}

One of our previous research focuses on the same concept as this thesis, which also utilizes U-Plane traffic for congestion solution and anomaly detection tasks in large-scale 5G IIoT environment \cite{shi2024uplane}. This scheme relies heavily on Aether ROC's unique capability to perform RAN management based on predefined groups. The groups can be defined at both the device and application levels, enabling the two-level congestion solution. It is a unique capability compared to RIC, which can only perform device-level management.

The proposed scheme aims to increase the anomaly detection system's performance by releasing Aether Onramp's potential for two-level RAN management.

\subsection{Aether Onramp Testbed Compared to O-RAN-based Conventional Ones}
Most existing studies use the O-RAN architecture for KPI monitoring and network management. However, O-RAN-based RAN management methods, such as network slicing, can only manage radio resources at the device (UE) level and lack the capability to manage resources at the application level.
In contrast, Aether Onramp\cite{aetheronramp} is a fully open-source software-defined radio network system. It features comprehensive KPI monitoring and RAN management capabilities. Aether Onramp uses Mega-patch to read in a preset network configuration, including predefined device groups, applications, network slicing, etc. Compared to O-RAN, another significant advantage of Aether Onramp is its ability to manage resources on an application basis. Its capability of managing network slices at both the device and application levels enables a two-level RAN management approach.

\subsection{System Overview and Strenth}

\begin{figure}
    \centering
    \includegraphics[width=1\linewidth]{Img/fig_shi/overview_mod.png}
    \caption{the Overview of the Proposed System}
    \label{fig:overview}
\end{figure}


\begin{figure}
    \centering
    \includegraphics[width=1\linewidth]{Img/fig_shi/process_mod.png}
    \caption{the Overview of the Anomaly Detection Scheme}
    \label{fig:adsystem}
\end{figure}

As shown in Fig. \ref{fig:overview}, this scheme consists of three main parts: a Local 5G RAN system based on Aether Onramp, a cluster of MQTT-based IIoT devices, and the anomaly detection core, including the machine learning model and I/O modules. The RAN system is the backbone of the IIoT network. In this scheme, the Aether Onramp 5G Core network (5GC) is hosted as a series of Kubernetes containers, gNodeB (gNB), and UEs provided by UERANSIM are deployed on separate machines. Similar to the proposed system of this thesis, the MQTT protocol has also been selected for the implementation scenario of this scheme. 

\subsubsection{Summary}

Although Aether ROC possesses a unique two-level RAN management capability, it is inferior compared to O-RAN RIC in terms of responsiveness. This is because the Mega-Patch design of Aether ROC was not originally intended for real-time adjustment of slice settings during network operations. Instead, it was designed to rapidly initialize network structures post-deployment. As a result, Aether ROC is incapable of achieving near-real-time control.

In contrast, one of RIC's core design objectives was to enable near-real-time management, which gives it a significant advantage in responsiveness. Furthermore, Aether ROC is not a component of the O-RAN standard, which limits the rapid adoption of Aether-based solutions within the standardized O-RAN environment.

On the other hand, RIC and other standardized O-RAN components, thanks to their standardized development, are compatible with nearly all RAN systems built on the same architecture. While Aether ROC offers the unique two-level management capabilities, its limitations in responsiveness and standardization pose significant challenges to its broader applicability.
\chapter{Proposed System Design}
\label{chap:design}



\section{Implementation Scenario}
The investigated scenario of the proposed method assumes the following:
Within a factory-owned local 5G-IIoT environment, sensor and controller devices (i.e., UEs) connect to a local 5G system, adopting MQ Telemetry Transport (MQTT) protocol for data transmission. A machine learning processing node is also placed to host and provide necessary processing capability for the following parts: RIC for RAN management and a machine learning model for anomaly detection tasks. Examples of anomaly traffic include the generation of pressure sensor alarms and malicious inquiries. Machine learning is deployed in a centralized way since the low computing power and processing capability of IIoT devices cannot afford distributed learning computational burden.
The assumed anomaly traffic includes two types: out-of-bound false pressure readings emitted from the pressure sensors and simulated malicious inquiries from and towards false or unknown destinations.
\section{Proposed System Architecture}
This thesis proposes a congestion solution system for large-scale IIoT to address the outlined challenges based on non-supervised clustering device classification.  

\begin{figure}[h]

    \centering
    \includegraphics[width=1\linewidth]{Img/fig_shi/initnew.png}

    \caption{Initial State of the Proposed System}
    \label{fig:ps_init}

\end{figure}

\begin{figure}[h]

    \centering
    \includegraphics[width=1\linewidth]{Img/fig_shi/alarmnew.png}
    \caption{Anomaly Detection State during Congestion}
    \label{fig:ps_alert}

\end{figure}

\begin{figure}[h]

    \centering
    \includegraphics[width=1\linewidth]{Img/fig_shi/finishnew.png}

    \caption{Congestion Solution Method Activation State}
    \label{fig:ps_finish}
\end{figure}

\subsection{Overview of the System}
As shown in Fig. \ref{fig:ps_init}, the proposed system constantly collects U-Plane traffic using Tshark, a traffic packet capture and protocol analyzer tool. The collected data undergoes preprocessing to become training data for the unsupervised clustering device classification model. Device classification is performed every hour to learn traffic patterns for each device. The system will maintain this phase as long as there is no sign of congestion. If a device's classification result changes during this phase, it will be adopted as a normal situation. This scenario is common when new devices are added to the factory's IIoT network or a certain device is temporarily modified for maintenance. This mechanism eliminates false alarms during routine maintenance and other situations. The congestion detection is performed using KPI monitor xAPP, which collects C-Plane KPIs including throughput and latency.


Fig. \ref{fig:ps_alert} illustrates the system when congestion is detected. The proposed system immediately performs anomaly device detection as the RIC xAPP detects congestion in the RAN. The proposed method is designed to maintain devices that follow their normal traffic pattern work as usual. In contrast, the anomaly devices are moved to another gNB by RIC traffic steering xAPP. The devices having their traffic patterns changed are identified as anomalies.

Alongside with anomaly device detection, a backup gNB is activated. The anomaly device list is sent to the RIC, and Traffic Steering RIC xAPP steers all anomaly devices on the list from gNB1 (the gNB to which normal devices are connected) to the backup gNB to reduce the congestion at gNB1. The state where the congestion resolution method is activated is illustrated in Fig. \ref{fig:ps_finish}. When the overall throughput decreases to a certain level that will not cause congestion on gNB1, all the anomaly devices will be steered back to gNB1, the reserved gNB is shut down, and the system returns to its initial state. This approach effectively prevents congestion from affecting normal devices, ensuring the stability and reliability of the IIoT network.

\section{Anomaly Detection Scheme}

\subsection{Data Collection and Preprocessing}
The data collection module collects U-Plane traffic data and decrypts Transport Layer Security (TLS) encoded traffic. The proposed system uses Tshark to collect U-Plane traffic, including parameters from Open Systems Interconnection (OSI) layer 2 to layer 7. Traffic data collection is performed continuously for online learning.
Let the number of IoT devices (i.e., UEs) be \(N\). The UEs are denoted as \(U = \{ 
U_1, U_2, ..., U_N \}\), connecting with the gNB. 

The captured traffic for  \(U_n\) at time slot \(t\) is expressed as \(U{_{n}}^{(t)}\) , each batched \texttt{pcap} files \(B_i\) for online learning can be expressed as below: 

\begin{equation}
B_i = \{ U_n^{(t)} \mid 0 < n \leq N, i < t \leq 2i \}
\end{equation}

When the system starts, the initial timeslot is defined as \(0\), with an interval between batches \(i\). The batched data is stored as \texttt{pcap} files for online learning. The decryption of TLS-encoded traffic is performed by using the TLS encryption key shared from the MQTT Broker in advance, and decrypted traffic data is processed to create a training dataset. A maximum timeslot window is selected during the deployment. When the batch data count reaches max value, the oldest batch is removed from the dataset when a new batch is collected.

\subsection{K-Means with Silhouette Index-based k-value estimation}

In the proposed method, K-Means\cite{kmeans} is utilized for unsupervised clustering, and silhouette index\cite{silhouette} is used for cluster (class) count estimation. Although the K-means clustering is not the fastest method with top efficiency, it is the simplest implementation for a proof of concept.

In the proposed system, each data point indicates an individual MQTT session, a cluster is a specific type of communication. The data points included in the cluster are traffic sessions classified as belonging to this traffic class. The K-Means algorithm performs unsupervised clustering based on Euclidean distance between data point \(U_n^{(t)}\), then updates the centroids of clusters \(c_j\) to its Euclidean center. This progress is repeated until the squared Euclidean distance \(||U{_{n}}^{(t)} - c_j||^2\) is minimized. 

To determine the optimal cluster count of K-Means, the proposed method selects the silhouette index to estimate the K value.
The silhouette index evaluates how similar an object is to its own cluster compared to others. The formula gives the silhouette score for a single sample:

\begin{equation}
    s = \frac{(b-a)}{max(a,b)}
\end{equation}

Where a is the mean distance between a sample and all other points in the same cluster, and b is the smallest mean distance of the sample to all points in any other cluster of which the sample is not a part. The silhouette value \(s\) is calculated for all \(\left \{\ k\ |\ 1\leq k \leq n \ \right \}\), the best k value among all is selected to present the estimated cluster count.

\subsection{Traffic Classification and Anomaly Detection}


\begin{algorithm}[]
\centering
\caption{Traffic Classification and Anomaly Detection}
\begin{algorithmic}[3] % Enable line numbering
\REQUIRE Perform KMeans clustering on \(B_i\) to obtain cluster labels \(L\)
\REQUIRE Congestion indicator \(Congest\)
\STATE Initialize congestion indicator \(Congest\)
\FORALL{device \({U_n} \in U\)}
      \IF{\(c_{U_n}^{i} \neq c_{U_n}^{(i-1)}\)}
      \STATE Select action based on \(Congest\) indicator
        \IF{\(Congest = 1\)}
        \STATE Label \({U_n}\) as anomaly device
        \ELSE
        \STATE Update \(c_{U_n}^{(i-1)}\) to the value of \(c_{U_n}^{i}\)
        \ENDIF
      \ENDIF
\ENDFOR
\end{algorithmic}
\end{algorithm}

\begin{algorithm}[ht]
\caption{Phase 1: Device-Based Detection}
\begin{algorithmic}
\REQUIRE Device Groups \(D\), Applications \(A\), and 
Registration Mapping \(R = \{ a \rightarrow d \mid a \in A, d \in D \}\)
\ENSURE Anomalies List \(E\)

\FOR{each device \(d \in D\)}
    \FOR{each application \(a \in A\) on device \(d\)}
        \IF{\(d \notin f(a)\)}
            \STATE Mark \(a\) on \(d\) as anomaly \(e_a^d\)
            \STATE Append \(e_a^d\) to set \(E\)
        \ENDIF
    \ENDFOR
\ENDFOR
\RETURN Appended Anomalies List \(E\)
\end{algorithmic}
\end{algorithm}

\vspace{-4mm}

\begin{algorithm}[hb]
\caption{Phase 2: Application-Based Detection}
\begin{algorithmic}
\REQUIRE Device Groups \(D\), Applications \(A\), and Anomalies List \(E\)
\ENSURE Complete Anomalies List \(E\)

\FOR{each device group \(g \in D\)}
    \FOR{each subset \(g' \subseteq d\) such that \(d \in f(a)\)}
        \STATE Collect subset \(S_{g,a}\) of \(a\) on \(g\)
        \STATE Apply Unsupervised Isolation Forest on \(S_{g,a}\)
        \STATE Detect anomalies in \(S_{g,a}\)
        \FOR{each anomaly \(e_a^d \in S_{g,a}\)}
            \STATE Append \(e_a^d\) to set \(E\)
        \ENDFOR
    \ENDFOR
\ENDFOR
\RETURN Detected anomalies \(E\)
\end{algorithmic}
\end{algorithm}


In the proposed method, unsupervised clustering methods are used for traffic classification. Since unsupervised methods do not require direct correspondence between clusters and specific communication types (device types, attack methods, etc.), the proposed method creates and maintains a table including the device's identifier and the identifier of the cluster it belongs to. The cluster tag of device \(U_n\) in batch \(i\) is expressed as \(c_{U_n}^{i}\). In particular, for newly detected device \(U_x\), the \(c_{U_x}^{(i-1)}\) is treated as \texttt{null}. The congestion indicator \(Congest\) is a Boolean value set to 1 when congestion is detected and set to 0 if not. Algorithm 1 shows the process of classification and anomaly detection. The proposed method utilizes online learning to adapt to changes in traffic patterns if there is no congestion. Suppose the communication pattern of an existing device has changed (due to situations such as a device under maintenance or a new device being installed), but the system is not congested currently. In that case, the change will be recorded without throwing an alarm. Algorithm 1 returns the anomaly device list for congestion resolution RIC xAPP. The RIC xAPP performs traffic steering for the anomaly devices, isolating them to the backup gNB to prevent congestion on the main gNB.

\begin{table}[]
\caption{MQTT Features Selected for the Classification}
\label{tab:types}
\resizebox{\columnwidth}{!}{%
\begin{tabular}{|c|ccc|}
\hline
\textbf{Device}             & \multicolumn{3}{c|}{\textbf{Features}}                                            \\ \cline{2-4} 
 & \multicolumn{1}{c|}{\textit{\textbf{Data type}}} & \multicolumn{1}{c|}{\textit{\textbf{Length}}} & \textit{\textbf{Pattern}} \\ \hline
Temperature sensor          & \multicolumn{1}{c|}{Float}            & \multicolumn{1}{c|}{Short}     & Periodic \\ \hline
Log file uploader           & \multicolumn{1}{c|}{File (Binary)}    & \multicolumn{1}{c|}{Very long} & Periodic \\ \hline
Valve controller            & \multicolumn{1}{c|}{ON/OFF (Boolean)} & \multicolumn{1}{c|}{Short}     & Burst    \\ \hline
Error log collector         & \multicolumn{1}{c|}{File (Binary)}    & \multicolumn{1}{c|}{Long}      & Burst    \\ \hline
Camera                      & \multicolumn{1}{c|}{Stream (Binary)}  & \multicolumn{1}{c|}{Very long} & Constant \\ \hline
Pressure alarm (Anomaly)    & \multicolumn{1}{c|}{Float}            & \multicolumn{1}{c|}{Short}     & Burst    \\ \hline
Malicious Enquiry (Anomaly) & \multicolumn{1}{c|}{String}           & \multicolumn{1}{c|}{Medium}    & Burst    \\ \hline
\end{tabular}%
}
\end{table}

\section{Summary}

In this chapter, we provide a brief overview of supervised learning and unsupervised learning, the two approaches in machine learning. We compare their training methods, dataset requirements, and other key aspects. We then examine their performance in terms of accuracy, adaptability to new classes, and training cost. Based on this comparison, we arrive at the following conclusions:

While supervised learning demonstrates superior accuracy in static environments (where no devices are added, removed, or modified) and offers the advantage of producing results in readable natural language tags, its limitations become apparent in Industrial IoT (IIoT) environments. Given that IIoT environments are dynamic—with evolving device configurations and constantly changing attack methods—unsupervised learning approaches provide outstanding benefits in terms of reduced training costs and improved adaptability to unknown anomalies and class shifts.

Additionally, we discuss online learning as a method that significantly improves the ability to address the inextricable interlinkage between the multiple phases of model training, testing, and verification. In today’s rapidly evolving environments, where these phases are tightly integrated, online learning offers a practical and effective way to adapt to new threats and operational changes in real-time.
\chapter{Evaluation}
\label{chap:eval}


\section{Experiment Platform Establishment}

\begin{figure*}[]
        \centering
        \includegraphics[width=1\linewidth]{Img/fig_shi/exp_env.PNG}
        \caption{Experiment Platform}
        \label{fig:image}   
\end{figure*}


An experiment platform is established to evaluate the proposed method. 
Fig \ref{fig:image} shows all parts of the experiment platform, including the following parts:  

\textbf{Local 5G Connected Traffic Generator PC} - A PC running Paho MQTT client-side scripts to generate the seven variations of MQTT-IIoT traffic. The traffic generator PC is connected to the laboratory's local 5G network by wire-connecting to a local 5G-enabled smartphone, providing wireless access as a 5G gateway.

\textbf{MQTT Broker PC} - A PC hosts EMQX \cite{EMQX} MQTT Broker and is connected to 5GC directly by wired connection. The traffic capture is also performed on this broker PC.

\textbf{5GC and gNB Unit} - The experiment took place in the laboratory's original software-based Local 5G system\cite{nakaolab}. The system includes a server hosting 5GC and gNB, with an antenna for wireless transmission.

As shown in Table \ref{tab:types}, five types of regular and two types of anomaly traffic are generated. The content of the traffic (e.g., temperature values emitted by temperature sensors) is created using a bounded random number generator.

\begin{table}[]
\caption{Selected Features: Decoded Method}
\label{proposed}
\resizebox{\columnwidth}{!}{%
\begin{tabular}{|c|cc|}
\hline
\textbf{Feature}  & \multicolumn{2}{c|}{\textbf{Description}}                                         \\ \cline{2-3} 
                  & \multicolumn{1}{c|}{\textit{\textbf{Type}}} & \textit{\textbf{Meaning}}           \\ \hline
MQTT ClientID.len & \multicolumn{1}{c|}{N/A}                    & ClientID Length                     \\ \hline
MQTT Topic.len    & \multicolumn{1}{c|}{N/A}                    & Topic Length                        \\ \hline
MQTT Msg.len      & \multicolumn{1}{c|}{Max}                    & Message Payload Length              \\ \cline{2-2}
                  & \multicolumn{1}{c|}{Min}                    &                                     \\ \cline{2-2}
                  & \multicolumn{1}{c|}{Avg}                    &                                     \\ \hline
Pkt interval      & \multicolumn{1}{c|}{Avg}                    & Interval time from previous message \\ \hline
Pkt count         & \multicolumn{1}{c|}{N/A}                    & Packet count during the connection  \\ \hline
\end{tabular}%
}
\end{table}

\begin{table}[]
\caption{Selected Features: Encoded Method}
\label{Legacy}
\resizebox{\columnwidth}{!}{%
\begin{tabular}{|c|cc|}
\hline
\textbf{Feature} & \multicolumn{2}{c|}{\textbf{Description}}                                         \\ \cline{2-3} 
                 & \multicolumn{1}{c|}{\textit{\textbf{Type}}} & \textit{\textbf{Meaning}}           \\ \hline
Pkt.len          & \multicolumn{1}{c|}{Max}                    & Packet length during the connection \\ \cline{2-2}
                 & \multicolumn{1}{c|}{Min}                    &                                     \\ \cline{2-2}
                 & \multicolumn{1}{c|}{Avg}                    &                                     \\ \hline
Pkt interval     & \multicolumn{1}{c|}{Avg}                    & Interval time from previous message \\ \hline
Pkt count        & \multicolumn{1}{c|}{N/A}                    & Packet count during the connection  \\ \hline
\end{tabular}%
}
\end{table}

\begin{table}[]
\caption{Selected Features: C-Plane-based Method}
\label{cplane}
\resizebox{\columnwidth}{!}{%
\begin{tabular}{|c|cc|}
\hline
\textbf{Feature} & \multicolumn{2}{c|}{\textbf{Description}}                                              \\ \cline{2-3} 
                 & \multicolumn{1}{c|}{\textit{\textbf{Type}}} & \textit{\textbf{Meaning}}                \\ \hline
Throughput       & \multicolumn{1}{c|}{Max}                    & Overall throughput during the connection \\ \hline
Pkt interval     & \multicolumn{1}{c|}{Avg}                    & Interval time from previous message      \\ \hline
\end{tabular}%
}
\end{table}

The time interval \(i\) between updates is set to 3600 seconds, which is the point of balance between accuracy and processing speed for our experiment scenario. Tshark is used for traffic acquisition, operating on the MQTT Broker server and capturing all MQTT traffic flow from simulated IIoT devices. The batched file is processed into three datasets according to the accessible features of all three methods. The three different sets of features are selected as below:

\textbf{Decoded Method} - shown in Table \ref{proposed}, the features are selected from decoded MQTT messages, including detailed features unique to the MQTT protocol. The packets are decoded by Tshark using the MQTT Broker's TLS certification and further processed to extract the MQTT-related features. Since decoded traffic is utilized, we can dig into MQTT packets to extract relevant features, enabling higher classification accuracy. In the cases of real-world production environments, the device IDs of the same device type are usually composed of a fixed header and a random string, and devices with different purposes tend to communicate over distinct Topics. Additionally, the messages sent back by sensors typically exhibit similar formats and consistent traffic patterns, as indicated by packet parameters. Therefore, we select the set of features presented in Table II for classification based on decoded traffic.

\textbf{Encoded method} - shown in Table \ref{Legacy}, the features are selected from non-protocol unique features, simulating a traditional method based on U-Plane traffic but not specified on any protocol. In practice, these are calculated from U-Plane packet information. Unlike the decoded method, we cannot dig into MQTT packets for more distinctive MQTT parameters. However, compared to C-Plane parameters, the packet parameters still provide some helpful information on classifying traffic patterns.

\textbf{C-Plane Method} - shown in Table \ref{cplane}, the features are limited to those available on the C-Plane KPI Monitor, simulating traditional C-Plane KPIs-based methods. These features are calculated into equivalent C-Plane KPIs, as the C-Plane-based method can only observe these KPIs and cannot observe U-Plane data directly.

To evaluate the unsupervised learning method, the true label category with the highest proportion within a cluster will be designated as that cluster's predicted label. To simulate the online learning process, the dataset is split evenly into 20 batches, each including 5\% of the dataset. For each batch, 5\% more data is fed into the clustering model. In the final batch, 100\% of all data is used in the clustering. The experiment is operated 10 times with different seeds for the traffic generator to generate different traffic data. 

\subsection{OAIC-based Simulated U-Plane Interface with RIC Tester}

Since the laboratory's Local 5G system does not support the O-RAN RIC controller and the U-Plane interface for interacting with RIC, we establish another testbed for the evaluation of U-Plane-based RIC controlflow. 

OAIC \cite{OAIC}, similar to the aforementioned Aether Onramp, is also a local 5G RAN testbed. Built on srsRAN \cite{srsran}, OAIC is particularly designed for the development and testing of machine learning-based RICs. It offers a fully open-source architecture, a library and toolsets for developing AI controllers (OAIC-C), and a framework for testing AI-based RICs (OAIC-T). 

For the needs of the proposed system, OAIC provides two unique tools: an open UPF interface inherited from its parent project, srsRAN, and a testing tool named E2-like for RIC validation.

The open UPF interface allows capturing all U-Plane traffic using tools like Wireshark, enabling the testing of the proposed U-Plane-based concept within the OAIC environment. This simulated U-Plane interface addresses the absence of a U-Plane interface in the O-RAN architecture, which is exactly our proposed interface.

E2-like, as part of OAIC’s testing framework (OAIC-T), simulates the functionality of the O-RAN E2 interface by providing RIC with equivalent data and receiving control commands from RIC. This enables testing and validation of RIC-based control flows within the OAIC environment. Since the proposed system in this thesis is primarily tested on our lab’s original local 5G RAN, OAIC is utilized specifically for capturing generated U-Plane traffic and conducting unit tests of RIC within OAIC’s test environment.
\chapter{Evaluation}
\label{chap:eval}


\section{Experiment Platform Establishment}

\begin{figure*}[]
        \centering
        \includegraphics[width=1\linewidth]{Img/fig_shi/exp_env.PNG}
        \caption{Experiment Platform}
        \label{fig:image}   
\end{figure*}


An experiment platform is established to evaluate the proposed method. 
Fig \ref{fig:image} shows all parts of the experiment platform, including the following parts:  

\textbf{Local 5G Connected Traffic Generator PC} - A PC running Paho MQTT client-side scripts to generate the seven variations of MQTT-IIoT traffic. The traffic generator PC is connected to the laboratory's local 5G network by wire-connecting to a local 5G-enabled smartphone, providing wireless access as a 5G gateway.

\textbf{MQTT Broker PC} - A PC hosts EMQX \cite{EMQX} MQTT Broker and is connected to 5GC directly by wired connection. The traffic capture is also performed on this broker PC.

\textbf{5GC and gNB Unit} - The experiment took place in the laboratory's original software-based Local 5G system\cite{nakaolab}. The system includes a server hosting 5GC and gNB, with an antenna for wireless transmission.

As shown in Table \ref{tab:types}, five types of regular and two types of anomaly traffic are generated. The content of the traffic (e.g., temperature values emitted by temperature sensors) is created using a bounded random number generator.

\begin{table}[]
\caption{Selected Features: Decoded Method}
\label{proposed}
\resizebox{\columnwidth}{!}{%
\begin{tabular}{|c|cc|}
\hline
\textbf{Feature}  & \multicolumn{2}{c|}{\textbf{Description}}                                         \\ \cline{2-3} 
                  & \multicolumn{1}{c|}{\textit{\textbf{Type}}} & \textit{\textbf{Meaning}}           \\ \hline
MQTT ClientID.len & \multicolumn{1}{c|}{N/A}                    & ClientID Length                     \\ \hline
MQTT Topic.len    & \multicolumn{1}{c|}{N/A}                    & Topic Length                        \\ \hline
MQTT Msg.len      & \multicolumn{1}{c|}{Max}                    & Message Payload Length              \\ \cline{2-2}
                  & \multicolumn{1}{c|}{Min}                    &                                     \\ \cline{2-2}
                  & \multicolumn{1}{c|}{Avg}                    &                                     \\ \hline
Pkt interval      & \multicolumn{1}{c|}{Avg}                    & Interval time from previous message \\ \hline
Pkt count         & \multicolumn{1}{c|}{N/A}                    & Packet count during the connection  \\ \hline
\end{tabular}%
}
\end{table}

\begin{table}[]
\caption{Selected Features: Encoded Method}
\label{Legacy}
\resizebox{\columnwidth}{!}{%
\begin{tabular}{|c|cc|}
\hline
\textbf{Feature} & \multicolumn{2}{c|}{\textbf{Description}}                                         \\ \cline{2-3} 
                 & \multicolumn{1}{c|}{\textit{\textbf{Type}}} & \textit{\textbf{Meaning}}           \\ \hline
Pkt.len          & \multicolumn{1}{c|}{Max}                    & Packet length during the connection \\ \cline{2-2}
                 & \multicolumn{1}{c|}{Min}                    &                                     \\ \cline{2-2}
                 & \multicolumn{1}{c|}{Avg}                    &                                     \\ \hline
Pkt interval     & \multicolumn{1}{c|}{Avg}                    & Interval time from previous message \\ \hline
Pkt count        & \multicolumn{1}{c|}{N/A}                    & Packet count during the connection  \\ \hline
\end{tabular}%
}
\end{table}

\begin{table}[]
\caption{Selected Features: C-Plane-based Method}
\label{cplane}
\resizebox{\columnwidth}{!}{%
\begin{tabular}{|c|cc|}
\hline
\textbf{Feature} & \multicolumn{2}{c|}{\textbf{Description}}                                              \\ \cline{2-3} 
                 & \multicolumn{1}{c|}{\textit{\textbf{Type}}} & \textit{\textbf{Meaning}}                \\ \hline
Throughput       & \multicolumn{1}{c|}{Max}                    & Overall throughput during the connection \\ \hline
Pkt interval     & \multicolumn{1}{c|}{Avg}                    & Interval time from previous message      \\ \hline
\end{tabular}%
}
\end{table}

The time interval \(i\) between updates is set to 3600 seconds, which is the point of balance between accuracy and processing speed for our experiment scenario. Tshark is used for traffic acquisition, operating on the MQTT Broker server and capturing all MQTT traffic flow from simulated IIoT devices. The batched file is processed into three datasets according to the accessible features of all three methods. The three different sets of features are selected as below:

\textbf{Decoded Method} - shown in Table \ref{proposed}, the features are selected from decoded MQTT messages, including detailed features unique to the MQTT protocol. The packets are decoded by Tshark using the MQTT Broker's TLS certification and further processed to extract the MQTT-related features. Since decoded traffic is utilized, we can dig into MQTT packets to extract relevant features, enabling higher classification accuracy. In the cases of real-world production environments, the device IDs of the same device type are usually composed of a fixed header and a random string, and devices with different purposes tend to communicate over distinct Topics. Additionally, the messages sent back by sensors typically exhibit similar formats and consistent traffic patterns, as indicated by packet parameters. Therefore, we select the set of features presented in Table II for classification based on decoded traffic.

\textbf{Encoded method} - shown in Table \ref{Legacy}, the features are selected from non-protocol unique features, simulating a traditional method based on U-Plane traffic but not specified on any protocol. In practice, these are calculated from U-Plane packet information. Unlike the decoded method, we cannot dig into MQTT packets for more distinctive MQTT parameters. However, compared to C-Plane parameters, the packet parameters still provide some helpful information on classifying traffic patterns.

\textbf{C-Plane Method} - shown in Table \ref{cplane}, the features are limited to those available on the C-Plane KPI Monitor, simulating traditional C-Plane KPIs-based methods. These features are calculated into equivalent C-Plane KPIs, as the C-Plane-based method can only observe these KPIs and cannot observe U-Plane data directly.

To evaluate the unsupervised learning method, the true label category with the highest proportion within a cluster will be designated as that cluster's predicted label. To simulate the online learning process, the dataset is split evenly into 20 batches, each including 5\% of the dataset. For each batch, 5\% more data is fed into the clustering model. In the final batch, 100\% of all data is used in the clustering. The experiment is operated 10 times with different seeds for the traffic generator to generate different traffic data. 

\subsection{OAIC-based Simulated U-Plane Interface with RIC Tester}

Since the laboratory's Local 5G system does not support the O-RAN RIC controller and the U-Plane interface for interacting with RIC, we establish another testbed for the evaluation of U-Plane-based RIC controlflow. 

OAIC \cite{OAIC}, similar to the aforementioned Aether Onramp, is also a local 5G RAN testbed. Built on srsRAN \cite{srsran}, OAIC is particularly designed for the development and testing of machine learning-based RICs. It offers a fully open-source architecture, a library and toolsets for developing AI controllers (OAIC-C), and a framework for testing AI-based RICs (OAIC-T). 

For the needs of the proposed system, OAIC provides two unique tools: an open UPF interface inherited from its parent project, srsRAN, and a testing tool named E2-like for RIC validation.

The open UPF interface allows capturing all U-Plane traffic using tools like Wireshark, enabling the testing of the proposed U-Plane-based concept within the OAIC environment. This simulated U-Plane interface addresses the absence of a U-Plane interface in the O-RAN architecture, which is exactly our proposed interface.

E2-like, as part of OAIC’s testing framework (OAIC-T), simulates the functionality of the O-RAN E2 interface by providing RIC with equivalent data and receiving control commands from RIC. This enables testing and validation of RIC-based control flows within the OAIC environment. Since the proposed system in this thesis is primarily tested on our lab’s original local 5G RAN, OAIC is utilized specifically for capturing generated U-Plane traffic and conducting unit tests of RIC within OAIC’s test environment.
\chapter{Evaluation}
\label{chap:eval}

\input{Chapters/Evaluation/Sections/Testbed}
\input{Chapters/Evaluation/Sections/Evaluation}
\chapter{Conclusion}
\label{chap:conclusion}

This thesis proposes an anomaly device detection and congestion avoidance system based on U-Plane data utilizing RIC. This paper is concluded as follow:\par

\vspace{5mm}


\textbf{Proposed a U-Plane-based Anomaly Detection and Congestion Avoidance System}
    
    \begin{itemize}
        \item Developed a novel system leveraging unsupervised online learning to detect anomaly traffic and mitigate network congestion.
        \item Introduced a mechanism for isolating anomalous devices from the primary network using RAN Intelligent Controller (RIC) xAPPs.
    \end{itemize}

\textbf{Introducing the Use of U-Plane Traffic Data in Classification}

        \begin{itemize}
        \item Unlike traditional methods relying solely on C-Plane KPIs, this system integrates U-Plane traffic parameters for detailed analysis, enabling accurate anomaly detection and device classification.
        \item Demonstrated that U-Plane data significantly enhances classification accuracy compared to C-Plane-only methods.
    \end{itemize}

\textbf{Unsupervised Online Learning for IIoT Anomaly Detection}
        \begin{itemize}
        \item Utilized an unsupervised clustering algorithm with online learning to adapt to dynamic IIoT environments, addressing the limitations of supervised learning such as dependency on labeled datasets and inflexibility with unknown classes.
        \item Achieved a classification accuracy of 94\%, outperforming traditional methods with only 63\% accuracy.
    \end{itemize}
    
\textbf{Development of a Comprehensive Experimental Platform}
    
        \begin{itemize}
        \item Established a simulation and experimental setup based on a laboratory-developed 5G network to evaluate the proposed system.
        \item Conducted extensive experiments to validate the system's performance in realistic IIoT scenarios, including dynamic traffic patterns and varying device types.
    \end{itemize}
    
\textbf{Enhanced Scalability and Adaptability}
    
        \begin{itemize}
        \item Addressed the scalability challenges of IIoT systems by ensuring the system can operate efficiently in large-scale environments.
        \item Demonstrated the system's adaptability to evolving threats and new device types without significant manual intervention.
    \end{itemize}
\chapter{Future Work}
\label{chap:future work}

For future work, we first plan to integrate the RIC xAPPs developed in this experiment with the main system to ensure system integrity. 

We will then deploy the proposed system on the O-RAN-based large-scale digital twin IoT testbed, Colosseum \cite{polese2024colosseum}, for further experimentation. This experiment aims to validate the accuracy and other evaluation metrics of the proposed method in an environment closer to real-world usage. Additionally, we will assess the system's performance in a large-scale device swarm scenario to evaluate its scalability. Key aspects to be tested include the latency of classification and anomaly detection and the system’s ability to address congestion in large-scale IoT environments effectively. In this experiment, we select the K-means with silhouette score for the clustering, but the combination is for the proof of concept and is not the top-efficiency selection. Therefore in the large-scale experiment, we will compare different unsupervised machine-learning methods, evaluate the efficiency and scalability, to find out the best equilibrium in accuracy and processing speed.

Furthermore, as one of the core arguments of this study, we demonstrated that leveraging U-Plane parameters for device classification and anomaly detection in RAN management significantly improves detection accuracy compared to traditional methods based on C-Plane KPIs. Therefore, one of the long-term objectives of this research is to advise the research community of the importance of U-Plane parameters in RAN management. To this end, we propose adding a new interface connecting the User Plane Function (UPF) with the RIC within the O-RAN architecture. This interface would provide RIC with additional parameters, significantly increasing the set of parameters available for RIC-based anomaly detection and introducing new possibilities for the research community.

In this experiment, the U-Plane parameters are captured in the OAIC testbed using its U-Plane interface. However, as the U-Plane interface is not formally part of the O-RAN architecture, this experiment is not considered performed under O-RAN standardization. Thus, as long as the U-Plane interface is not integrated into the O-RAN architecture, this system cannot be considered a standardized component for RIC, as it requires an additional source of parameters from UPF. Therefore, our future work will also include further investigation and advancement of C-Plane-based solutions. Without U-Plane parameters, it is more challenging to provide the same accuracy.

%-------------------
\bibliographystyle{ieeetr} % 参考文献
\bibliography{myref} %
%-------------------
\end{document}