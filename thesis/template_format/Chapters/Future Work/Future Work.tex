\chapter{Future Work}
\label{chap:future work}

For future work, we first plan to integrate the RIC xAPPs developed in this experiment with the main system to ensure system integrity. 

We will then deploy the proposed system on the O-RAN-based large-scale digital twin IoT testbed, Colosseum \cite{polese2024colosseum}, for further experimentation. This experiment aims to validate the accuracy and other evaluation metrics of the proposed method in an environment closer to real-world usage. Additionally, we will assess the system's performance in a large-scale device swarm scenario to evaluate its scalability. Key aspects to be tested include the latency of classification and anomaly detection and the system’s ability to address congestion in large-scale IoT environments effectively. In this experiment, we select the K-means with silhouette score for the clustering, but the combination is for the proof of concept and is not the top-efficiency selection. Therefore in the large-scale experiment, we will compare different unsupervised machine-learning methods, evaluate the efficiency and scalability, to find out the best equilibrium in accuracy and processing speed.

Furthermore, as one of the core arguments of this study, we demonstrated that leveraging U-Plane parameters for device classification and anomaly detection in RAN management significantly improves detection accuracy compared to traditional methods based on C-Plane KPIs. Therefore, one of the long-term objectives of this research is to advise the research community of the importance of U-Plane parameters in RAN management. To this end, we propose adding a new interface connecting the User Plane Function (UPF) with the RIC within the O-RAN architecture. This interface would provide RIC with additional parameters, significantly increasing the set of parameters available for RIC-based anomaly detection and introducing new possibilities for the research community.

In this experiment, the U-Plane parameters are captured in the OAIC testbed using its U-Plane interface. However, as the U-Plane interface is not formally part of the O-RAN architecture, this experiment is not considered performed under O-RAN standardization. Thus, as long as the U-Plane interface is not integrated into the O-RAN architecture, this system cannot be considered a standardized component for RIC, as it requires an additional source of parameters from UPF. Therefore, our future work will also include further investigation and advancement of C-Plane-based solutions. Without U-Plane parameters, it is more challenging to provide the same accuracy.