\section{Unsupervised Learning}

Although supervised learning is widely regarded as the most accurate method in static environments, it has several limitations. For instance, its lack of flexibility means that the training dataset must include all devices and anomaly types that might appear in the system; otherwise, the model cannot function properly. Additionally, anomaly detection models often encounter unknown anomalies in real-world applications, where supervised learning-based models become ineffective.
Furthermore, supervised learning requires a labeled dataset for training, meaning each dataset used for updates must be manually labeled. This significantly increases the system's training and operational costs.
In contrast, unsupervised learning has emerged as a more cost-effective option in IIoT systems, which process large amounts of data on a large-scale swarm of devices. Since IIoT environments have a more straightforward communication structure and fewer protocols than non-IoT environments, unsupervised learning is better suited to these less complicated environments. Additionally, unsupervised approaches do not rely on labeled datasets for training, making them a more economical choice compared to supervised ones in the IIoT environment. 