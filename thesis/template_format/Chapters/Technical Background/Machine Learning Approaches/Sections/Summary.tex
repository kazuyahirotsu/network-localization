\section{Summary}

In this chapter, we provide a brief overview of supervised learning and unsupervised learning, the two approaches in machine learning. We compare their training methods, dataset requirements, and other key aspects. We then examine their performance in terms of accuracy, adaptability to new classes, and training cost. Based on this comparison, we arrive at the following conclusions:

While supervised learning demonstrates superior accuracy in static environments (where no devices are added, removed, or modified) and offers the advantage of producing results in readable natural language tags, its limitations become apparent in Industrial IoT (IIoT) environments. Given that IIoT environments are dynamic—with evolving device configurations and constantly changing attack methods—unsupervised learning approaches provide outstanding benefits in terms of reduced training costs and improved adaptability to unknown anomalies and class shifts.

Additionally, we discuss online learning as a method that significantly improves the ability to address the inextricable interlinkage between the multiple phases of model training, testing, and verification. In today’s rapidly evolving environments, where these phases are tightly integrated, online learning offers a practical and effective way to adapt to new threats and operational changes in real-time.