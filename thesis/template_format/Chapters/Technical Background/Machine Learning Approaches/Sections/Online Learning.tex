\section{Online Learning}
Commonly, previous studies \cite{bakar2015activity} \cite{chatterjee2022iot} \cite{sisinni2018industrial} have utilized pre-trained machine learning models for tasks such as performance anomaly detection in production environments. However, production environments are often not static; they evolve with changes in production demands or the installation of new industrial equipment. When the types and quantities of equipment, or IIoT devices, increase, resulting in new traffic patterns, pre-prepared recognition models based on static datasets are likely to experience a decrease in accuracy or even complete failure. For instance, newly added device types may be identified as anomalies because they are not registered in the recognition model. Therefore, we need a machine learning approach—online learning—that can better adapt to the continuously changing numbers and types of devices and support adaptive training to compensate for these changes in the environment.

One critical challenge in the practical application of machine learning models for RAN management is the inextricable interlinkage between multiple phases: model training, testing, and verification, which are performed continuously as part of a continuous flow. This implies that when new threats or failure types emerge, a system designed for real-world deployment must possess the capability to process data on-site, capture anomalies, evaluate the findings, and integrate the updated system into operation. This highlights a significant shift from traditional approaches in earlier studies, where pre-trained models—often optimized through cherry-picking and fine-tuned for peak accuracy—were deployed into production environments with the expectation of delivering long-term reliability. Such static, one-time deployments are no longer viable in modern RAN management's dynamic and rapidly evolving context. Online learning is considered one of the must-have solutions to this situation and could be implemented in more solutions when the RIC-based anomaly detection method is deployed in real smart factories or other application scenarios.

