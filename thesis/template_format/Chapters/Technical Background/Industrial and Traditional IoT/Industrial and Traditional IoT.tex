\section{Difference between Industrial and Traditional IoT}
\label{chap:IIoT}

Previous research on topics including anomaly detection and congestion prevention has primarily focused on traditional IoT environments, such as Smart Home IoT \cite{alaa2017review} \cite{bakar2015activity} \cite{ramapatruni2019anomaly}. However, this thesis examines the industrial IoT (IIoT) environment, which differs in many ways from traditional Smart Home IoT \cite{sisinni2018industrial}. This chapter provides an overview of both industrial and traditional IoT environments, highlighting their key differences. It also examines how these differences influence the design and effectiveness of anomaly detection and congestion prevention solutions.

\subsection{Smart Home IoT}

Smart Home IoT, as the name suggests, refers to network solutions deployed in residential or indoor environments to support a wide range of smart home devices \cite{alaa2017review}. Typical smart home devices include cameras, sensors, voice-active assistants like Amazon Alexa, smart TVs, computers, and mobile phones. In principle, any device within a residential system that connects to the home's wireless network (usually Wi-Fi) falls under this category. As a result, the key characteristics of Smart Home IoT are the diversity of device types, significant differences in communication structures, and a relatively small number of devices installed with limited data transmission volumes (compared to Industrial IoT, or IIoT). However, due to the privacy-sensitive nature of residential environments, the demand for anomaly detection in smart home settings is much higher than in IIoT.

Given these characteristics, anomaly detection solutions tailored for Smart Home IoT have their unique features \cite{chatterjee2022iot}. Smart Home IoT does not typically require robust congestion prevention solutions since the number of devices installed is relatively small and high-volume communications rarely occur simultaneously, compared with IIoT. However, the high priority on privacy protection in smart home environments creates a strong demand for both external intrusion detection and internal anomaly detection systems. Consequently, a considerable count of research has focused on anomaly detection in Smart Home IoT \cite{bakar2015activity}. The diverse communication protocols used in smart home environments also present significant challenges for anomaly detection tasks. Unlike IIoT, where detecting anomalies among multiple similar devices is common, anomaly detection in Smart Home IoT is often device-specific. In this context, traditional rule-based firewall architectures usually outperform machine learning approaches in terms of accuracy, as computational resources are relatively sufficient. This is because machine learning methods struggle to deliver high precision in such environments.

\subsection{Industrial IoT}
Industrial IoT (IIoT) refers to IoT network systems deployed in industrial environments such as smart factories. Unlike Smart Home IoT, which involves fewer devices with diverse functionalities, IIoT is designed to support a large number of industrial devices that are varied in multiple types \cite{sisinni2018industrial}. These devices, primarily sensors and controllers, are distributed across factory facilities. In contrast to the diverse communication protocols employed in Smart Home IoT, IIoT environments typically use fewer or even a single protocol for data transmission. This standardization is largely due to the fact that all sensors and controllers are integrated into a centralized control system. While the importance of external intrusion and anomaly detection is equally critical in IIoT and Smart Home IoT environments, the massive number of devices and the inclusion of time-sensitive components—such as controllers requiring millisecond-level synchronization—make IIoT systems more susceptible to congestion. Furthermore, because of the large scale of devices and their sensitivity to interference and latency, Wi-Fi, with its lower interference resistance and limited communication capacity, is rarely used in IIoT environments. Instead, private local 5G networks are often adopted for connectivity.

These unique characteristics mean that congestion prevention and anomaly detection solutions in IIoT environments differ significantly from those developed for Smart Home IoT \cite{ramapatruni2019anomaly}. Solutions for IIoT must be capable of handling the massive communication generated by high-concurrency operations among numerous devices, as this is a fundamental requirement for IIoT systems. Such demands necessitate high processing capabilities and often lead to the adoption of control-plane key performance indicator (C-Plane KPI) analysis rather than user-plane (U-Plane) packet capture for anomaly detection. C-Plane KPI analysis offers considerable advantages in terms of acquisition and processing speed compared to U-Plane traffic analysis. Combined with the RAN Intelligent Controller (RIC), it enables near-real-time control loops, allowing the system to respond swiftly to anomalies—an essential feature in IIoT environments, where minimizing downtime is critical.

Additionally, the large number of identical devices and the uniformity of protocols make machine learning an optimal approach for IIoT control loops. Supervised learning-based methods can quickly and accurately identify labeled anomalies, while unsupervised learning-based methods are highly effective in detecting unknown anomalies. Their flexibility and adaptability allow them to localize anomalous devices in real time, facilitating emergency responses and supporting subsequent manual decision-making.

\subsubsection{MQTT Protocol}
This subsection introduces the MQ Telemetry Transport (MQTT) protocol \cite{soni2017survey}, a lightweight protocol designed to meet the needs of IoT systems. With appropriate client implementations, MQTT enables high concurrency while stressing less on servers compared to conventional protocols \cite{rao2015implementing}\cite{naik2017choice}.

Conventional protocols (e.g., TCP, UDP) were originally designed for internet-based communication. In such environments, factors like transmission efficiency (the ratio of payload to the overall packet size) are not always the highest priority. Additionally, traditional protocols exhibit several characteristics that are not fully suited to Industrial IoT applications. To address these limitations, the MQTT protocol was developed. It is specifically tailored for IoT environments, effectively meeting the transmission demands and communication constraints of IIoT devices. As the comparison between MQTT and other IoT protocols is shown in Fig. \ref{fig:protocol-compare}, the key features of the MQTT protocol are as follows:
\begin{description}
    \item[Lightweight Protocol] In Industrial IoT environments, where a large swarm of devices generates high volumes of traffic, lightweight communication protocols are crucial for alleviating communication pressure. MQTT is based on the conventional TCP protocol but optimizes several unnecessary components in IoT contexts. This streamlining significantly improves transmission efficiency, reducing the overhead associated with communication.
    \item[Centralized Broker-Client Structure] Edge IoT devices often have limited processing capabilities, making the computational burden on clients a significant challenge. To address this issue, MQTT employs a centralized broker-client architecture. In this structure, all communication is initiated by clients and processed centrally by the MQTT broker before being distributed. This design shifts the computational burden to the broker, significantly reducing the processing demands on clients. Consequently, the protocol can operate efficiently on low-performance edge IoT devices, lowering the hardware requirements for connected devices.
    \item[Topic-based Multithread] MQTT utilizes a topic-based multithreading model for communication management. Each topic includes multiple publishers and subscribers. When a publisher pushes information to the broker, the broker automatically broadcasts the message to all subscribers of that topic. Each edge IoT device can execute multiple threads simultaneously, allowing it to listen to one topic while publishing information to another, and vice versa. This multithreading mechanism provides flexibility and scalability for communication in IoT environments.
\end{description}

\begin{figure}
    \centering
    \includegraphics[width=1\linewidth]{Img/related/protocol-cpmpare.png}
    \caption{A comparison between MQTT and Other Lightweight IoT Protocols. from \cite{naik2017choice}}
    \label{fig:protocol-compare}
\end{figure}

\subsection{Summary}

This section provides an overview of the difference between traditional IoT environments (using smart home IoT as a sample) and Industrial IoT. This chapter also compares the anomaly detection methods used in conventional and industrial IoT, based on their respective different environments. The comparison is summarized in the Table. \ref{tab:IoT-compare}. Additionally, this chapter briefly introduces the MQTT protocol, one of the most used lightweight elastic protocols for the IIoT environment.

\begin{table}[h]
\caption{Comparison between Conventional Smart Home and Industrial IoT}
\label{tab:IoT-compare}
\resizebox{\columnwidth}{!}{%
\begin{tabular}{|l|l|l|}
\hline
                              & Conventional IoT & Industrial IoT \\ \hline
Device Quantity               & Low              & High           \\ \hline
Device Class Diversity        & High             & Low            \\ \hline
Device Count in Class         & Low              & High           \\ \hline
Latency Tolerance             & High             & Extremely Low  \\ \hline
Protocol Diversity            & High             & Low (Unified)  \\ \hline
Access Method                 & Mainly Wi-Fi     & Local 5G       \\ \hline
Anomaly Detection Requirement & Strict           & Relatively Low \\ \hline
Managemnt Basis               & Device-based     & Class-based    \\ \hline
\end{tabular}%
}
\end{table}