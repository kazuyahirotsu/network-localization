\section{Contributions}

Compared to related work, this thesis proposes a U-Plane-based congestion avoidance RIC xAPP scheme for large-scale 5G-IIoT systems. The proposed scheme combines high-accuracy U-Plane-based anomaly detection with intelligent C-Plane network management enabled by RIC. As a result, the proposed method achieves a 94\% classification accuracy, while the accuracy of the traditional method is 63\%. The contributions and advantages of this proposed method are summarized as follows:
\begin{description}
    \item[Provide Detailed Parameters for Anomaly Detection] A RIC-based U-Plane anomaly detection and congestion solution system is proposed, capable of collecting highly detailed raw U-Plane traffic data and isolating anomaly devices. This structure has enabled RIC's RAN management capability by providing high-accuracy anomaly detection results based on U-Plane data.
    \item[Cost-friendly, Flexible Anomaly Detection Scheme] We propose a traffic classification and anomaly detection method based on unsupervised online learning. The proposed method is designed to collect datasets for the training. The unsupervised online learning approach enables flexible classification, which can detect unknown anomalies and adapt to changes like device reorganization.
    \item[Propose Autonomous Solution utilizing RIC] We introduce an autonomous scheme to capture U-Plane traffic and extract the parameters for machine learning-based anomaly detection. Labeling work is not required as we use unsupervised learning, leading to an autonomous data collection module that automatically collects data independently. Then, we utilize RIC to perform the congestion solution to offload the anomaly devices from the main network. The whole control flow is fully autonomous, without the need for man-in-the-circuit instructions.
    \item[Setting Up Testbed for the evaluation] We establish the simulation and experimental platform to evaluate the effectiveness of the proposed method. Therefore, we can evaluate the capability of the proposed method by putting the system into a simulated IIoT-RAN environment, which mimics the real-world IIoT traffic.

\end{description}
The result proves that the proposed method is effective in performing anomaly detection in our simulated IIoT environment, with a significant advancement of 31\% in overall accuracy compared to the traditional method.