\section{Organization}
This thesis proposes an unsupervised machine learning-based adaptive anomaly detection and congestion avoidance for large-scale 5G-Industrial IoT systems.

The thesis chapters are organized as follows.

\textbf{Chapter 1:} First, We introduce the background of the research, which is the need for RAN management methods in the concept of 5G-IIoT systems. Then, we elaborate on the problem statement, thesis statement, and objectives. We then provide an overview of related works about machine learning-based RAN management and its utilization in IoT environments. 

\textbf{Chapter 2:} In this chapter, we provide a more profound introduction to the background of this study: the Industrial IoT (IIoT) environment. Given that most prior research has primarily focused on traditional IoT environments, such as Smart Home IoT, we also provide an overview of the characteristics of conventional IoT environments. Furthermore, we conduct a comparative analysis between traditional IoT and IIoT, highlighting their key differences and discussing how these differences influence the choice of RAN management strategies. 

We also introduce the private local 5G system, which serves as the RAN backbone for Industrial IoT environments. The discussion included a comparison between private local 5G systems and WiFi, highlighting their respective advantages and limitations in IIoT applications. Additionally, we provide an overview of the core components of the private 5G system, including the 5G Core Network (5GC), the Radio Access Network (RAN), and the RAN Intelligent Controller (RIC), detailing their roles and functionalities.

\textbf{Chapter 3:} This chapter includes a survey of related research in the area of anomaly detection for IIoT, and also consists of our previous research, a U-Plane-based two-level approach for the same topic.

\textbf{Chapter 4:} This chapter presents our proposed system design, including the implementation scenario, the structure of the anomaly detection scheme, the machine learning algorithms we select for the proposed method, and the detailed workflow of the scheme.

\textbf{Chapter 5:} This chapter covers the experiment and evaluation of the proposed system. By deploying the proposed system in a simulated Industrial IoT environment built on our laboratory's original 5G network, we evaluate the accuracy of the proposed system.

\textbf{Chapter 6:} Conclusion of this thesis.

\textbf{Chapter 7:} Outlook of the future work.
