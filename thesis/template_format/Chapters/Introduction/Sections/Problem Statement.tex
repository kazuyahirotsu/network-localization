\section{Problem Statement}

This thesis aims to develop an anomaly detection-based congestion solution scheme for local 5G-based IIoT system. For the task of congestion solution, three main modules should be included in the scheme - data collection, anomaly detection, and RAN management. 

\begin{description}
    \item[Lacking detailed features] The first problem of this research is the difficulty of providing detailed parameters for anomaly detection by the data collection module. As utilized in current methods, C-Plane KPIs \cite{c-plane_parameter} are the conventional choice for parameter collection. However, the problem is the C-Plane KPI only includes indicators such as throughput, connection count, end-to-end latency, and slice-related indicators. These indicators are designed to observe the workload and monitor the performance of the RAN at the network layer \cite{osi_model} and are not designed for anomaly detection tasks. Therefore, only utilizing C-Plane parameters results in a lack of application layer parameters, which are critical to anomaly detection tasks in the IIoT environment. On the other hand, U-Plane traffic is the raw traffic generated from devices, including application layer parameters. U-Plane traffic parameters captured at the data link layer are widely utilized for anomaly detection tasks and have been proven to be effective \cite{chatterjee2022iot} by providing in-detail information from all captured packets. The most popular U-Plane traffic capture tool, Wireshark \cite{wireshark}, provides thousands of available in-depth parameters from the application layer to below. Therefore, this research needs to enable the utilization of U-Plane traffic in the data collection module instead of C-Plane KPIs in existing research. 
    \item[High Training Cost and Inflexible Classification] The second problem of this research is finding a low-cost, flexible anomaly detection method for the proposed scheme. In previous research, supervised machine learning provides high accuracy for anomaly detection tasks over predefined classes \cite{supervised}. However, supervised learning cannot detect undefined anomalies and requires a labeled dataset for every new batch of training data. As a result, a training dataset must be labeled accurately and include all possible types of anomalies. This results in a high training cost and a lack of flexibility, especially for a large-scale IIoT system, which continuously generates a large amount of data flow, making manual labeling work difficult. Unsupervised learning is considered a low-cost approach compared with supervised one. Since unsupervised learning does not rely on labeled datasets to perform classification and anomaly detection, it has lower operational costs than supervised learning per training batch. Also, as the unsupervised approach dramatically cuts the training cost, online learning can be applied to form a flexible solution that adapts to changes in the IIoT system. However, the unsupervised learning approach also requires a more detailed dataset to maintain accuracy, as it relies on clustering for the classification. Also, the online learning approach requires a constant data collection scheme. As a result, the problem for the proposed scheme is to provide a constant flow of detailed data to feed the anomaly detection module.
    \item[Lacking U-Plane-based Autonomous Solution] The research's final problem is establishing an autonomous scheme to perform the congestion solution based on anomaly detection results. Current studies investigate autonomous solutions to mitigate the congestion mainly by utilizing Radio Access Network (RAN) Intelligent Controller (RIC)\cite{RIC}. RIC provides access to RAN's control plane, enabling machine learning-based methods for network management. As defined in the O-RAN architecture, RIC connects with multiple interfaces for C-Plane KPI information collection. However, the current O-RAN architecture does not have an interface for capturing U-Plane traffic. Therefore, a U-Plane parameter-based anomaly detection scheme cannot interconnect with RIC in the current situation. This research has to design an interface-like structure for the data collection module to collect U-Plane parameters.  
\end{description}



