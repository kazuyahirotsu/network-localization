\section{Background}

Alongside the evolution of Industry 4.0, the era of the Internet of Things (IoT)  results in an ever-expanding scale of interconnecting devices. Also known as the Industrial Cyber-Physic System (CPS), factories utilize a tremendous number of sensors and controller devices to sustain their normal function \cite{large_number_sensors}. The interconnecting large-scale swarm of Industrial IoT (IIoT) devices generates and transmits tremendous traffic, inflicting a heavy burden on the backbone network system and may cause congestion. Since sensor devices tend to rely on wireless network connections, the backbone Radio Access Network (RAN) constantly faces the challenge of congestion.


\begin{figure}[hb]
    \centering
    \includegraphics[width=0.9\linewidth]{Img/fig_shi/security_issues.png}
    \caption{Classifcation of security issues faced by the 5G-IIoT smart
factory (From \cite{congestion_reason_and_threat})}
    \label{fig:background1}
\end{figure}

\begin{figure}[hb]
    \centering
    \includegraphics[width=0.9\linewidth]{Img/fig_shi/securityissue2.png}
    \caption{Security and privacy issues in 5G-IIoT smart factories (From \cite{congestion_reason_and_threat})}
    \label{fig:background2}
\end{figure}

Nevertheless, large-scale IIoT systems are especially vulnerable to network congestion issues due to their extensive scale. Congestion can be caused by spikes in transmission demand, malfunctions, malicious attacks, etc.
Research indicates that congestion in large-scale IIoT systems significantly impacts production safety, poses a severe risk to safety-critical systems \cite{congestion_reason_and_threat}, and even makes safety-critical mechanisms become unfunctional. The sudden decrease in transfer rate and increase in latency leads to transmission failures and device desynchronization, causing the production lines to halt. Therefore, an anomaly detection scheme is considered a critical component for the robustness of large-scale IIoT RAN systems. 
