\section{U-Plane Anomaly Detection}

在过往研究已经涉及过的设备分类/异常检测任务中,U-Plane data由于包含real traffic,也就是说,并非只是某种特征的indicator,而有广泛的应用。传统方法往往使用wireshark等工具,直接在客户端处captureU-Plane通讯。以此方式获取的通讯提供了大量有助于实现设备/应用分类的特征量,尤其是OSI模型中较高层的特征量。例如,TCP封包的flag和其他众多重要参数可以在此获取。作为一种IoT环境下常用的轻量级通讯协议,MQTT协议的通讯内容也属于application layer,并且也可以被wireshark等工具获取。客户端ID,Topic ID和QoS Flag等等重要的参数均可被获取,由此使得针对MQTT设备的异常检测得以实现。

(此处过往研究介绍)

在5G RAN系统中,U-Plane data通常在UPF处取得。我们在((哪一章节??))介绍过5G RAN系统中 5G Corenet的具体结构,UPF是其中负责承载和沟通U-Plane数据的重要功能模块。然而,在目前的5G RAN management中,O-RAN架构并未定义有关UPF的interface,这给基于U-Plane的网络管理和控制带来了诸多不便。