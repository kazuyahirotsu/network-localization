\chapter{Conclusion}
\label{chap:conclusion}

This thesis proposes an anomaly device detection and congestion avoidance system based on U-Plane data utilizing RIC. This paper is concluded as follow:\par

\vspace{5mm}


\textbf{Proposed a U-Plane-based Anomaly Detection and Congestion Avoidance System}
    
    \begin{itemize}
        \item Developed a novel system leveraging unsupervised online learning to detect anomaly traffic and mitigate network congestion.
        \item Introduced a mechanism for isolating anomalous devices from the primary network using RAN Intelligent Controller (RIC) xAPPs.
    \end{itemize}

\textbf{Introducing the Use of U-Plane Traffic Data in Classification}

        \begin{itemize}
        \item Unlike traditional methods relying solely on C-Plane KPIs, this system integrates U-Plane traffic parameters for detailed analysis, enabling accurate anomaly detection and device classification.
        \item Demonstrated that U-Plane data significantly enhances classification accuracy compared to C-Plane-only methods.
    \end{itemize}

\textbf{Unsupervised Online Learning for IIoT Anomaly Detection}
        \begin{itemize}
        \item Utilized an unsupervised clustering algorithm with online learning to adapt to dynamic IIoT environments, addressing the limitations of supervised learning such as dependency on labeled datasets and inflexibility with unknown classes.
        \item Achieved a classification accuracy of 94\%, outperforming traditional methods with only 63\% accuracy.
    \end{itemize}
    
\textbf{Development of a Comprehensive Experimental Platform}
    
        \begin{itemize}
        \item Established a simulation and experimental setup based on a laboratory-developed 5G network to evaluate the proposed system.
        \item Conducted extensive experiments to validate the system's performance in realistic IIoT scenarios, including dynamic traffic patterns and varying device types.
    \end{itemize}
    
\textbf{Enhanced Scalability and Adaptability}
    
        \begin{itemize}
        \item Addressed the scalability challenges of IIoT systems by ensuring the system can operate efficiently in large-scale environments.
        \item Demonstrated the system's adaptability to evolving threats and new device types without significant manual intervention.
    \end{itemize}